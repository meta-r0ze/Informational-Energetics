\crefalias{section}{appendix}
\section{The \boldmath{\texorpdfstring{$E_8$}{E8}} Root Inventory and State Partition}
\label{app:root_inventory_and_state_partition}

To demonstrate that the Standard Model particle content is not an arbitrary selection, we provide a rigorous accounting of the 248 root vectors of $E_8$. We show that the 48 fundamental fermions are the unique subset of the root system that satisfies the Persistence Principle under the maximal subgroup decomposition $E_8 \supset E_6 \times SU(3)$.

\subsection{The Fundamental Decomposition}

The $E_8$ Lie algebra (dimension 248) decomposes under $E_6 \times SU(3)$ as:

\begin{equation}
\mathbf{248} = (\mathbf{78}, \mathbf{1}) \oplus (\mathbf{1}, \mathbf{8}) \oplus (\mathbf{27}, \mathbf{3}) \oplus (\overline{\mathbf{27}}, \overline{\mathbf{3}})
\end{equation}

This partition assigns every root vector in the lattice to a specific physical sector based on its transformation properties:

\begin{itemize}
    \item \textbf{The Gauge Core $(\mathbf{78}, \mathbf{1})$:} Contains the unification field generators, including the Standard Model gauge group $SU(3) \times SU(2) \times U(1)$.
    \item \textbf{The Flavor Core $(\mathbf{1}, \mathbf{8})$:} Contains the generators of the generational symmetry (identifying the 3-fold generation index).
    \item \textbf{Matter Precursors $(\mathbf{27}, \mathbf{3})$:} Contains three families of matter, transforming as the fundamental representation of $E_6$.
    \item \textbf{Mirror Precursors $(\overline{\mathbf{27}}, \overline{\mathbf{3}})$:} Contains three families of conjugate mirror matter.
\end{itemize}

\subsubsection{The Persistence Filter (Decomposition of the 27)}

The 81 roots identified as ``Matter'' consist of three copies of the $\mathbf{27}$ representation of $E_6$. Under the geometric descent to the Standard Model via $SO(10)$, the $\mathbf{27}$ decomposes as:
\begin{equation}
    \mathbf{27} \to \mathbf{16} \oplus \mathbf{10} \oplus \mathbf{1}
\end{equation}
The Persistence Principle, specifically the requirement for a stable topological boundary ($\chi=2$) established in Section IV.D, acts as a geometric filter on these sub-representations. We posit that for a particle to be a persistent, localized entity (a knot in the vacuum), its corresponding representation must be able to support this topological charge.

\begin{enumerate}
    \item \textbf{The $\mathbf{16}$ (Chiral Spinor):} Contains the Standard Model fermions. Spinor representations possess a unique topological nature, exemplified by the $720^\circ$ rotation required to return to the original state (the "spinor double cover"). We propose that this inherent topological complexity is the necessary algebraic foundation to support a stable, non-trivial knot structure ($\chi=2$). These states can "tie themselves" into persistent particles. \textbf{Status: Retained.}

    \item \textbf{The $\mathbf{10}$ (Vector):} Contains vector-like quarks and leptons. Vector representations lack the topological complexity of spinors. They transform like simple vectors and do not have the double-cover property. We argue that this makes them topologically trivial; they cannot support a stable, closed boundary. Any attempt to form a localized particle from these states would correspond to a trivial topology ($\chi=0$), which would immediately unravel and dissipate into the vacuum. \textbf{Status: Filtered.}

    \item \textbf{The $\mathbf{1}$ (Singlet):} Contains the sterile neutrino modulus. As a singlet, this state is, by definition, decoupled from the gauge interactions tethering it to the manifold. The boundary of a particle is defined by its interaction field lines. A singlet, having no gauge charges, cannot produce the field lines necessary to define a boundary that separates "inside" from "outside." It is informationally untethered, cannot satisfy Gauss's Law, and cannot form a persistent boundary of any kind. \textbf{Status: Filtered.}
\end{enumerate}
Therefore, the topological requirement for a persistent particle acts as a sharp filter, selecting only the spinor representation of $SO(10)$ as a viable candidate for low-energy matter.

\subsection{The Balance Sheet (\texorpdfstring{248 $\to$ 48}{248to48})}

We provide the complete ledger of the $E_8$ roots, classifying them into \textbf{Persistent (SM)} and \textbf{Filtered (Decoupled)} categories.

\paragraph{The Excluded Sectors (200 Roots)}
The following roots are geometrically prohibited from forming low-energy states:
\begin{itemize}
    \item \textbf{Mirror Sector (81 Roots):} The $\overline{\mathbf{27}}$ sector is orthogonal to the chiral projection $P_L$ defined in Eq.~(4). It constitutes the ``Dark Sector'' which may interact gravitationally but lacks the geometric alignment to couple to the Standard Model gauge fields.
    \item \textbf{Vector/Sterile Sector (33 Roots):} The $\mathbf{10}$ and $\mathbf{1}$ components of the $\mathbf{27}$ (11 roots per generation $\times$ 3 generations).
    \item \textbf{Heavy Gauge Sector (66 Roots):} The components of the $\mathbf{78}$ excluding the 12 Standard Model bosons. These correspond to GUT-scale X/Y bosons which acquire Planck-scale masses.
    \item \textbf{Flavor Generators (8 Roots):} Internal symmetry operators that do not manifest as propagating fields.
    \item \textbf{Total Excluded:} $81 + 33 + 66 + 8 = 188$ roots.
\end{itemize}

\paragraph{The Persistent Sector (60 Roots)}
The remaining roots satisfy all geometric invariants ($\nu=16, \chi=2, D=4$):
\begin{itemize}
    \item \textbf{Standard Model Bosons (12 Roots):} The photon, $W^\pm$, $Z$, and 8 gluons.
    \item \textbf{Standard Model Fermions (48 Roots):}
    \begin{itemize}
        \item 3 Generations $\times$ 16 Chiral States.
        \item ($\nu = 16$ states = 6 Quarks + 2 Leptons + Antiparticles).
    \end{itemize}
\end{itemize}

\textbf{Conclusion:} The 48 fermions of the Standard Model are not an arbitrary collection; they are the unique subset of $E_8$ roots that possess both Chiral Persistence ($P_L$) and Topological Stability ($\chi=2$). All other roots are strictly filtered by the geometry of the projection.