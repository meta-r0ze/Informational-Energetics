\section{Theoretical Context: Informational Energetics}
\label{sec:IE}
A subset of complex adaptive systems, persistent systems must optimize for persistence against entropy. 

This framework unifies insights from non-equilibrium thermodynamics, algorithmic information theory, and robust control theory, bridging them with empirical principles from evolutionary biology, computational neuroscience, and high-energy physics, and applying them through the practical lenses of institutional economics, quantitative finance, and reliability engineering.

% TODO citations to add, so many possiblities, options...
% Non-equilibrium thermodynamics → cite Prigogine or England
% Algorithmic information theory → cite Kolmogorov, Chaitin, or Solomonoff
% Robust control theory → cite Doyle or Csete & Doyle (2002) on biological robustness
% Evolutionary biology → cite Kauffman (self-organization) or England (dissipation-driven adaptation)

\subsection{The Axiom of Persistence}

The fundamental imperative of persistence is to maximize the duration of existence against environmental entropy. \textit{This can be viewed as a generalization of the Anthropic Principle, shifting the focus from the conditions required for observers to the more fundamental conditions required for any structure to exist at all.}
We formalize this as follows:

\begin{mdframed}[linewidth=1pt,linecolor=black,backgroundcolor=gray!10]
    \textbf{Persistence Principle}: A system maximizes its total Persistence Value ($P$) by minimizing its \textbf{Entropic Action} ($S_\Phi$) relative to its structural complexity.
\end{mdframed}

While open systems (like biology) persist by maximizing energy intake, the vacuum is a closed system. Therefore, the persistence criterion shifts: rather than maximizing intake, the vacuum must \textbf{minimize information loss}.

To satisfy the Axiom, any persistent entity must implement a specific architecture comprising four structural pillars for information management, plus the thermodynamic overhead of operation on its substrate.

\subsection{The Structural Pillars}

The set of structural requirements that make up all \textbf{Persistence Systems} ($Z_{IE}$).

\begin{equation}
\label{eq:IE_pillars}
\begin{split}
{} & Z_{IE} = \\ 
&  \underbrace{\Delta E}_{\text{Capacity}}
+ \underbrace{\Delta I}_{\text{Identity}}
- \underbrace{MI}_{\text{Efficiency}}
- \underbrace{G}_{\text{Stability}}
+ \underbrace{T}_{\text{Overhead}}
+ \underbrace{PM}_{\text{Margin}}
\end{split}
\end{equation}

\noindent The six components represent the universal structural requirements of persistence. To manifest themselves, these abstract requirements must map to specific features of a system. These six components represent the minimal complete set: fewer leaves the system unable to persist; additional components reduce to combinations of these. Positive terms represent entropic costs; negative terms represent efficiency gains that reduce the persistence burden.

These structural pillars constitute the universal architecture of any persistent entity:

\begin{enumerate}
    \item \textbf{The Energy Vessel ($\Delta E$):} \textit{The Capacity.} 
    The physical infrastructure required to acquire resources and perform work. It defines the maximum bandwidth, storage limit, and energy throughput of the system. Without a vessel, the system lacks the agency to act.

    \item \textbf{The Information Model ($\Delta I$):} \textit{The Identity.} 
    The internal logic or topological structure that distinguishes the system from the environment. It functions as the predictive engine, encoding the system's configuration to reduce environmental uncertainty. Without a model, the system acts blindly.
    
    \item \textbf{The Coordination Protocol ($MI$):} \textit{The Efficiency.} 
    The communicative glue regulating the flow between the Vessel and the Model. It ensures coherence and minimizes the entropic loss of signal transmission. Without a protocol, the system fragments into isolated parts.
    
    \item \textbf{The Stabilizing Governor ($G$):} \textit{The Stability.} 
    The constraint mechanism that prevents unbounded divergence. It enforces the operational boundaries necessary to maintain structural integrity against internal pressure. Without a governor, the system consumes itself or explodes.
    
    \item \textbf{The Temporal Cost ($T$):} \textit{The Overhead.} 
    The entropic cost of state transitions. It represents the irreversible energy expenditure required to update the system's configuration, enforcing the arrow of time.
    
    \item \textbf{The Persistence Margin ($PM$):} \textit{The Resolution Floor.} 
    The buffer for existence. It represents the minimum resolution limit required to distinguish a signal from thermal background noise, or the reserve capacity required to survive fluctuations.
\end{enumerate}

The mapping of Information Energetics (IE) to specific physical domains is inherently adaptive, reflecting the unique constraints of each persistent system. While the present work \textit{demonstrates} this on the fundamental substrate of the vacuum and the geometric derivation of the fine-structure constant ($\alpha^{-1}$), the IE framework is functionally isomorphic to all systems that persist, a Rosetta Stone for persistent systems.

\subsection{The Universal Architecture: Derivation via Control Theory}
\label{sec:PillarCompleteness}

To rigorously establish the necessity and sufficiency of the structural pillars, we model the persistent entity not as a static object, but as a \textbf{Dynamic Control System} regulating its internal state against environmental fluctuations (Entropy).

In Control Theory, the stability of any regulatory loop is governed by well-defined mathematical requirements (e.g., Lyapunov Stability, Nyquist Criterion). We assert that the six pillars of Informational Energetics are the thermodynamic isomorphisms of these non-negotiable control components.

\subsubsection{The Isomorphism to Robust Control}

Any feedback control loop requires a specific set of functional blocks to operate. We map these standard engineering components to the pillars of persistence:

\begin{description}
    \item[\textbf{Capacity ($\Delta E$):}] \textbf{The Plant / Actuator} The ability to perform work to correct state deviations.
    \item[\textbf{Identity ($\Delta I$):}] \textbf{Reference Signal (Setpoint)} The definition of the ``target state'' ($S_0$) distinct from the environment.
    \item[\textbf{Protocol ($MI$):}] \textbf{Feedback Loop / Sensor} The transmission channel connecting the Plant state to the Controller.
    \item[\textbf{Governor ($G$):}] \textbf{Stability Criterion} The negative feedback gain required to prevent unbounded divergence.
    \item[\textbf{Cost ($T$):}] \textbf{Processing Latency} The irreversible time delay between sensing and actuation.
    \item[\textbf{Margin ($PM$):}] \textbf{Gain/Phase Margin} The buffer against parameter drift or external shock.
\end{description}

This mapping allows us to leverage established theorems in control theory to argue for completeness. A control loop without a Plant cannot act ($\Delta E$); without a Setpoint, it has no goal ($\Delta I$); without Feedback, it runs open-loop ($MI$); without Stability Criteria, it oscillates ($G$).

\subsubsection{Proof of Necessity (The Failure Modes)}

We demonstrate the necessity of the set $\mathbb{P} = \{\Delta E, \Delta I, MI, G, T, PM\}$ by analyzing the \textbf{Counterfactual Failure Mode} of a system $S$ where exactly one pillar is removed ($S' = \mathbb{P} \setminus \{x\}$). In every case, the expected lifetime of the system $\tau$ tends to zero.

\begin{enumerate}
    \item \textbf{Removal of Capacity ($\Delta E \to 0$): The Starvation Mode.}
    Consider a system with a perfect model and protocol but zero energy capacity. It detects the entropic threat but cannot perform the work $W$ required to counter it.
    \textit{Result:} $\tau \to 0$ due to thermodynamic equilibrium. \textit{The Ghost}.

    \item \textbf{Removal of Identity ($\Delta I \to 0$): The Dissolution Mode.}
    Consider a system with infinite energy but no definition of ``Self.'' The control loop lacks a Reference Signal. The Actuator fires randomly, increasing internal entropy rather than decreasing it.
    \textit{Result:} $\tau \to 0$ due to loss of boundary. \textit{The Cloud}.

    \item \textbf{Removal of Protocol ($MI \to 0$): The Decoherence Mode.}
    Consider a system where the Plant and Controller are severed. The Controller issues commands based on outdated state data; the Plant acts without instruction. The components become statistically independent.
    \textit{Result:} $\tau \to 0$ due to fragmentation. \textit{The Schism}.

    \item \textbf{Removal of Governor ($G \to 0$): The Divergence Mode.}
    Consider a system with positive feedback but no negative feedback constraints. Any fluctuation $\delta$ is amplified: $S_{t+1} = S_t(1 + \alpha)$. The system energy grows exponentially until it exceeds the structural limit of the substrate.
    \textit{Result:} $\tau \to 0$ due to structural rupture. \textit{The Explosion}.

    \item \textbf{Removal of Temporal Cost ($T \to 0$): The Zeno Mode.}
    Consider a system that attempts to update instantaneously. By the \textbf{Margolus-Levitin theorem}, a minimum time
\begin{equation}
    \Delta E \cdot \Delta t \geq \frac{h}{4} \quad 
\end{equation}
    (or specifically $\hbar \ln 2$ for information-theoretic limits) is required for any orthogonal state transition. A control loop with zero processing latency would require either infinite energy or zero state change. 

    \textbf{Result:} The system either cannot evolve (frozen) or violates energy conservation (physically impossible).

    \item \textbf{Removal of Margin ($PM \to 0$): The Fragility Mode.}
    Consider a system operating at the theoretical efficiency limit (Criticality). It persists only as long as the environmental variance $\sigma_{env} = 0$. In a real environment ($\sigma_{env} > 0$), the first fluctuation pushes the state outside the basin of attraction.
    \textit{Result:} $\tau \to \text{Random Variable}$ (Survival is probabilistic and brittle).
\end{enumerate}

\textbf{Conclusion:} Since the removal of any single component results in immediate or statistical termination, the set is \textbf{Minimally Necessary}.

To demonstrate sufficiency, we note that any stable control system requires exactly these components. Control theory provides no additional fundamental requirements beyond Plant, Setpoint, Feedback, Stability, Latency, and Robustness. Any proposed seventh pillar would necessarily decompose into combinations of these six primitive functions.

\subsection{Note on Systemic State: Quiescent Equilibrium}

Informational Energetics describes the lifecycle of systems through phases of Genesis, Expansion, Stasis, and Collapse via the mathematics of Logistic Maps/Bifurcation. It is important to note that the system derived here, the Vacuum (System 0), represents a specific thermodynamic state known as \textbf{Quiescent Equilibrium}.

Unlike biological or economic systems which are in a state of Expansion or Brittle Stasis, the fundamental laws of physics represent a system that has minimized its metabolic drag to the absolute theoretical floor ($S_\Phi \to 0$). The geometric invariants derived are not evolving parameters; they are the \textbf{Fixed Point Attractors} of the vacuum's self-optimization. The universe is not currently ``evolving'' new laws; it is persisting within the optimal solution.