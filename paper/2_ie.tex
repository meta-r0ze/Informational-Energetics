\section{Theoretical Context: Informational Energetics}
\label{sec:IE}
A subset of complex adaptive systems, persistent systems must optimize for persistence against entropy. 

This framework synthesizes insights from non-equilibrium thermodynamics, algorithmic information theory, and robust control theory, bridging them with empirical principles from evolutionary biology, computational neuroscience, and high-energy physics, and applying them through the practical lenses of institutional economics, quantitative finance, and reliability engineering.

Informational Energetics is a synthetic framework that models persistence by unifying principles from three domains:
Algorithmic Information Theory: To define the complexity and identity of a system.
Non-Equilibrium Thermodynamics: To define the entropic costs and energy flows required to maintain that identity.
Robust Control Theory: To define the governance and stability mechanisms that prevent systemic failure.
The bridge between these fields is the principle that information is physical (Landauer), meaning that all informational operations (computation, storage, error correction) have irreducible thermodynamic costs. IE, therefore, treats reality as a control system operating under thermodynamic constraints to minimize information loss.

% TODO citations to add, so many possiblities, options...
% Non-equilibrium thermodynamics → cite Prigogine or England
% Algorithmic information theory → cite Kolmogorov, Chaitin, or Solomonoff
% Robust control theory → cite Doyle or Csete & Doyle (2002) on biological robustness
% Evolutionary biology → cite Kauffman (self-organization) or England (dissipation-driven adaptation)

\subsection{The Axiom of Persistence}

The fundamental imperative of persistence is to maximize existence duration against environmental entropy. We formalize this as the \textbf{Persistence Principle}: the minimization of \textbf{Entropic Action ($S_\Phi$)} relative to structural complexity. This action represents the metabolic cost of maintaining a distinct identity. To satisfy this axiom, any persistent entity must implement a specific architecture comprising four structural pillars for information management, plus the thermodynamic overhead of operation.

The Principle of Persistence dictates that a system must minimize the loss of structural information. This creates a bridge between disciplines:
\begin{itemize}
    \item \textbf{To a Physicist:} This is the \textit{Principle of Least Action} applied to geometry.
    \item \textbf{To a Computer Scientist:} This is \textit{Loss Function Minimization} in a distributed network.
    \item \textbf{To a Biologist:} This is the minimization of \textit{Metabolic Drag} required to maintain homeostasis.
\end{itemize}

\subsection{The Structural Pillars}
The set of structural requirements that make up all \textbf{Persistence Systems} ($P$).

\begin{equation}
\label{eq:IE_pillars}
\begin{split}
{} & P = \\ 
&  \underbrace{\Delta E}_{\text{Capacity}}
+ \underbrace{\Delta I}_{\text{Identity}}
- \underbrace{MI}_{\text{Efficiency}}
- \underbrace{G}_{\text{Stability}}
+ \underbrace{T}_{\text{Overhead}}
+ \underbrace{PM}_{\text{Margin}}
\end{split}
\end{equation}

\noindent The six components represent the universal structural requirements of persistence. To manifest, these abstract requirements must map to specific features of a system. These six components represent the minimal complete set: fewer leaves the system unable to persist; additional components reduce to combinations of these. Positive terms represent entropic costs; negative terms represent efficiency gains that reduce the persistence burden.

These structural pillars constitute the universal architecture of any persistent entity:

\begin{enumerate}
    \item \textbf{The Energy Vessel ($\Delta E$):} \textit{The Capacity.} 
    The physical infrastructure required to acquire resources and perform work. It defines the maximum bandwidth, storage limit, or energy throughput of the system. Without a vessel, the system lacks the agency to act.

    \item \textbf{The Information Model ($\Delta I$):} \textit{The Identity.} 
    The internal logic or topological structure that distinguishes the system from the environment. It functions as the predictive engine, encoding the system's configuration to reduce environmental uncertainty. Without a model, the system acts blindly.
    
    \item \textbf{The Coordination Protocol ($MI$):} \textit{The Efficiency.} 
    The communicative glue regulating the flow between the Vessel and the Model. It ensures coherence and minimizes the entropic loss of signal transmission. Without a protocol, the system fragments into isolated parts.
    
    \item \textbf{The Stabilizing Governor ($G$):} \textit{The Stability.} 
    The constraint mechanism that prevents unbounded divergence. It enforces the operational boundaries necessary to maintain structural integrity against internal pressure. Without a governor, the system consumes itself or explodes.
    
    \item \textbf{The Temporal Cost ($T$):} \textit{The Overhead.} 
    The entropic cost of state transitions. It represents the irreversible energy expenditure required to update the system's configuration, enforcing the arrow of time.
    
    \item \textbf{The Persistence Margin ($PM$):} \textit{The Resolution Floor.} 
    The buffer for existence. It represents the minimum resolution limit required to distinguish a signal from thermal background noise, or the reserve capacity required to survive fluctuations.
\end{enumerate}

\subsection{Coordination Signals}

We distinguish \textbf{Persistent Systems}, which maintain a distinct identity ($\Delta E > 0$) against the environment, and \textbf{Coordination Signals}, which serve as the transient transmission mechanism for the Protocol ($MI$). A System exists in time; a Signal propagates through space.

\subsection{The Architectural Completeness of the Pillars}
\label{sec:PillarCompleteness}

The assertion that these six pillars constitute the universal architecture of persistence is not an arbitrary enumeration, but a claim of functional completeness. The framework posits that this set is both necessary and sufficient, forming a minimal, orthogonal basis for describing any system that maintains structural coherence against entropy over time.

This claim is rooted in the idea that to persist, any entity must provide answers to six fundamental and irreducible questions of existence. The pillars are the functional answers to these questions.

\begin{itemize}
    \item \textbf{Question of Substance:} \textit{What is the system made of? What are its resources?} \\
    \textbf{Answer:} The \textbf{Energy Vessel (Capacity, $\Delta E$)}. A system without substance or access to energy is a mere abstraction and cannot act or persist.

    \item \textbf{Question of Identity:} \textit{How is the system distinct from its environment? What defines it?} \\
    \textbf{Answer:} The \textbf{Information Model (Identity, $\Delta I$)}. A system without a defining identity is not a system at all.

    \item \textbf{Question of Coherence:} \textit{How do the system's parts communicate and work together?} \\
    \textbf{Answer:} The \textbf{Coordination Protocol (Protocol, $MI$)}. Without it the system becomes a collection of disconnected parts.

    \item \textbf{Question of Stability:} \textit{How does the system prevent its own disintegration or unbounded divergence?} \\
    \textbf{Answer:} The \textbf{Stabilizing Governor (Governor, $G$)}. A system without self-regulation is inherently unstable.

    \item \textbf{Question of Dynamics:} \textit{How does the system evolve or change state in an ordered manner?} \\
    \textbf{Answer:} The \textbf{Temporal Cost (Overhead, $T$)}. This cost enforces a well-defined arrow of time, preventing paradoxical or acausal evolution.

    \item \textbf{Question of Resilience:} \textit{How does the system survive environmental noise and fluctuations?} \\
    \textbf{Answer:} The \textbf{Persistence Margin (Margin, $PM$)}. A system with zero margin for error is fragile and will be destroyed by random chance.
\end{itemize}

\textbf{Necessity and Orthogonality:} The set is argued to be minimal because the removal of any one pillar renders a system incapable of long-term persistence. The pillars are functionally orthogonal; for example, the resources a system possesses (Capacity) are distinct from the rules of how those resources are used (Protocol), which are in turn distinct from the constraints that prevent overuse (Governor).

\textbf{Sufficiency:} The set is argued to be complete because any other complex function of a persistent system, such as adaptation or error-correction, can be described as a dynamic interplay of these six fundamental pillars. Error-correction, for instance, is a process that uses the Protocol to detect a deviation, the Governor to correct it, and the Margin to absorb the cost of the error.

Therefore, the six pillars are not a random list but are presented as a complete, non-negotiable architectural basis for existence. They form a coordinate system for the functional state space of any persistent entity.