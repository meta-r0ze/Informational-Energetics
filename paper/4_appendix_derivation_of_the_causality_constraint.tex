\crefalias{section}{appendix}
\section{Formal Derivation of the Causality Constraint}
\label{app:derivation_of_the_causality_constraint}

In the main text, we asserted that a causal projection requires the Fundamental Resonance $\Delta$ to be greater than the total state space $N=32$. This appendix provides the rigorous mathematical justification for that constraint. We first formally define the projection operators that separate the matter (chiral) and mirror (symmetric) sectors, prove their geometric independence, and then derive the non-aliasing condition that emerges from this structure.

\subsection{The Projection Operators}

The E\textsubscript{8} lattice embeds in $\mathbb{R}^8$. We define the projection from the 8D lattice space to the 4D spacetime manifold using the orthogonal decomposition inherent to the $E_8 \to D_4 \oplus D_4$ split. Let $x \in \mathbb{R}^8$ be a lattice vector. We define the Left-Chiral ($P_L$) and Right-Chiral ($P_R$) projection operators as:
\[
P_L(x) = \frac{1}{\sqrt{2}}(x_1 - x_2, x_3 - x_4, x_5 - x_6, x_7 - x_8)
\]
\[
P_R(x) = \frac{1}{\sqrt{2}}(x_1 + x_2, x_3 + x_4, x_5 + x_6, x_7 + x_8)
\]

\subsection{Orthogonality Proof}

For the projections to define distinct physical sectors (Matter vs. Mirror), they must be orthogonal. We compute the inner product. Since the basis vectors are orthogonal roots in the fundamental domain of E\textsubscript{8} ($|x_i|^2=1, x_i \cdot x_{i+1}=0$):
\begin{align*}
    P_L \cdot P_R &= \frac{1}{2} \sum_{i=1,3,5,7} (x_i - x_{i+1})(x_i + x_{i+1}) \\
    &= \frac{1}{2} \sum (x_i^2 - x_{i+1}^2) = \frac{1}{2} \sum (1 - 1) = 0
\end{align*}
Thus, $P_L \perp P_R$. The sectors are geometrically distinct.


\subsection{The Generalized Nyquist Limit}

The total information content of the lattice node is the sum of its chiral components.
\begin{itemize}
    \item Dimension of Left Sector: $\text{dim}(P_L) = 4$ spatial dimensions $\times$ 4 spinor components $= 16$ degrees of freedom ($\nu$). The image of $P_L$ is 4-dimensional. However, the chiral spinor representation on this 4D manifold has dimension $v = 16$ (the Weyl spinor of Spin(10)).
    \item Dimension of Right Sector: $\text{dim}(P_R) = 16$ degrees of freedom.
    \item \textbf{Total State Space: $N = 32$.}
\end{itemize}

For the lattice to project these $N=32$ degrees of freedom onto a single integer timeline (defined by the Fundamental Resonance $\Delta$) without \textbf{Aliasing}, we invoke the Pigeonhole Principle.

\paragraph{Theorem (Generalized Nyquist for Discrete Lattices):} Let a lattice have $N$ distinct state channels, and let $\Delta$ be the fundamental period. For a bijective projection from the $N$-channel state space onto the cyclic group $\mathbb{Z}/\Delta\mathbb{Z}$, we require $\Delta >= N$. If $\Delta < N$, then by the Pigeonhole Principle, at least two distinct channels must map to the same integer index in $\mathbb{Z}/\Delta\mathbb{Z}$. This creates an aliasing event, causing phase ambiguity between the Left and Right sectors.

Thus, the condition $\Delta > 32$ is a hard topological constraint of the projection.