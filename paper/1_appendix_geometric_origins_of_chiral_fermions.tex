\crefalias{section}{appendix}
\section{Geometric Origin of Chiral Fermions}
\label{app:geometric_origins_of_chiral_fermions}

Distler and Garibaldi provided a rigorous service to the field of physics by demonstrating that a continuous, algebraic embedding of the Standard Model within $E_8$ cannot accommodate chiral fermions without introducing unobserved ``mirror fermions.''

The $E_8$-Persistence framework circumvents this ``No-Go'' theorem by operating in a fundamentally different mathematical domain: \textbf{Geometric Projection} rather than Algebraic Embedding. As derived in \cref{sec:derivation_c}, the chirality of the Standard Model emerges not from the Lie algebra itself, but from the metric signature change required to project the Euclidean lattice onto a causal Lorentzian manifold.

This appendix summarizes that derivation specifically within the context of the Distler-Garibaldi constraints, demonstrating that our mechanism does not contradict their theorem but lies outside its assumptions.

\subsection{The Distler-Garibaldi Constraint}
The theorem holds for all \textbf{signature-preserving} embeddings, where the algebraic structure of the vacuum remains Euclidean or Compact ($8,0$). Under these conditions, the embedding is achiral; for every left-handed fermion, a right-handed mirror partner must exist to preserve the symmetry of the group.

\subsection{The Geometric Resolution}
Our framework utilizes \textbf{Metric Projection} involving a signature change ($8,0 \to 3,1$). By explicitly changing the metric signature, the projector $P_L$ acts as a \textbf{Chiral Filter}.

As detailed in Section \cref{sec:chiral_diode}, the introduction of the time-like metric axis forces the decoupling of the ``Mirror'' sector to preserve norm positivity. The mirror fermions required by $E_8$ symmetry still exist in the mathematical structure of the lattice, but they are rendered geometrically orthogonal to the observer's timeline. They are not erased; they are causally disconnected.

\begin{table}[h]
\centering
\caption{Comparison of Approaches}
\renewcommand{\arraystretch}{1.2}
\begin{tabular}{lcc}
\hline
\textbf{Feature} & \textbf{DG Analysis} & \textbf{This Work} \\
\hline
Method & Algebraic Embedding & Geometric Projection \\
Domain & Continuous Maps & Discrete Lattice \\
Signature & Preserved ($8,0$) & Changed ($8,0 \to 3,1$) \\
Mirror Fermions & Required (Problem) & Orthogonal (Decoupled) \\
\hline
\end{tabular}
\end{table}

Thus, the observed chiral asymmetry is not a violation of $E_8$ symmetry, but the geometric consequence of projecting a symmetric structure onto a causal timeline.