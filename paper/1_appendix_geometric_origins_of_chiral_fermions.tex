\crefalias{section}{appendix}
\section{Geometric Origin of Chiral Fermions}
\label{app:geometric_origins_of_chiral_fermions}
% This section isn't technically needed, everything is already elsewhere, but exist because of the DG paper and readers will look for it, thus is primarily cref's to the location.

The Standard Model exhibits a puzzling asymmetry: only left-handed fermions participate in weak interactions. In the E8-Persistence framework, this chirality emerges directly from the causal projection derived in \cref{sec:derivation_c}

\subsection{The Projection Mechanism}

The $E_8$ lattice is Euclidean with signature $(8,0)$. Physical spacetime is Lorentzian $(3,1)$. As derived in \cref{sec:spinor}, the requirement for complex information processing ($\nu=16$) mandates the Lorentzian metric.

\subsection{Why Projection Creates Chirality}
The signature change $(8,0) \to (1,3)$ fundamentally alters the spinor structure. As shown in \cref{sec:chiral_diode}

\begin{enumerate}
    \item The Lorentzian metric $g^{00} = -1$ creates norm-sign asymmetry between timelike and spacelike components.
    \item To preserve unitarity (norm positivity), the projection must couple exclusively to one chiral sector via the projector $P_L = \frac{1-\gamma^5}{2}$.
    \item The right-chiral sector remains geometrically orthogonal to the Lorentzian time direction (the mirror sector, \cref{sec:chiral_diode}).
\end{enumerate}

\subection{Distinction from Algebraic Embeddings}

This resolves the Distler-Garibaldi constraint. Their no-go theorem applies to signature-\emph{preserving} algebraic embeddings $\phi: G_{\text{SM}} \hookrightarrow E_8$ within fixed signature.

Our projection involves metric signature change and operates outside their assumptions:

\begin{center}
\begin{tabular}{lcc}
\hline
& Algebraic Embed. & Geometric Projection \\
\hline
Method & Continuous map & Discrete lattice map \\
Signature & Preserved & $(8,0) \to (1,3)$ \\
Mirror fermions & Forbidden & Orthogonal to timeline \\
\hline
\end{tabular}
\end{center}

Thus, chirality is the geometric consequence of causal projection, fully derived in \cref{sec:derivation_c}, not an additional assumption.