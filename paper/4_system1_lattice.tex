\section{System 1: The Lattice Substrate and the Geometric Invariants}
\label{sec:system1}

Having established that the vacuum must be a Discrete, Self-Dual, and Causal information processing substrate we now identify the one unique mathematical structure that satisfies these constraints.

Through a process of elimination we first determine the geometry of the lattice itself before projecting it into spacetime.

\subsection{The Lattice Selection (\texorpdfstring{$E_8$}{E8})}
\label{sec:derivation_a}
\label{sec:lattice_selection}

We now determine which of these permitted dimensions satisfies the remaining constraints of persistence.

\subsubsection{The \texorpdfstring{$n \equiv 0 \pmod 8$}{neq0mod8} Constraint}
\label{sec:kneserstheorem}
The requirement of an even, self-dual lattice is remarkably restrictive. A result from lattice theory, Kneser's Theorem \cite{kneser_klassenzahlen_1957}, states that even, self-dual lattices only exist in dimensions that are multiples of 8. This immediately eliminates the vast majority of dimensionalities leaving only: 

\[ D \in \{8, 16, 24, \dots\} \]
This strictly eliminates any lattice solution in dimensions $D<8$ including $D=4$ (Standard Relativity) or $D=10$ (Superstring Theory) as they cannot support a self-dual unitarity condition without auxiliary structures.

\subsubsection{The Principle of Minimum Configurational Entropy}
\label{sec:principleofminconfigentropy}
To minimize the Entropic Action of the substrate, the system must not only minimize geometric complexity (dimension) but also eliminate arbitrary selection parameters. We analyze the population of even, self-dual lattices permitted by Kneser's Theorem ($D = 8k$):

\begin{itemize}
    \item \textbf{D=8:} A unique solution exists (The $E_8$ lattice).
    \item \textbf{D=16:} Two distinct solutions exist ($E_8 \oplus E_8$ and $D_{16}^+$).
    \item \textbf{D=24:} Twenty-four distinct solutions exist (The Niemeier lattices).
\end{itemize}

A vacuum established at $D=16$ would possess an irreducible \textbf{Configurational Entropy} of $S_{config} = k_B \ln(2)$, representing the information required to distinguish between the two topological isomers. A vacuum at $D=24$ would have $S_{config} = k_B \ln(24)$.

The $E_8$ lattice ($D=8$) is the unique solution where $S_{config} = k_B \ln(1) = 0$. It is selected not merely for its low dimensionality, but because it is the only self-dual geometry that forms a deterministic ground state without inherent topological ambiguity.



\subsection{The Projection (\texorpdfstring{$D=4$}{D=4} and \texorpdfstring{$\nu=16$}{nu=16})}
\label{sec:derivation_b}
A static 8-dimensional block cannot process information; processing requires a flow (Input vs. Output). The system must break the $E_8$ symmetry to distinguish the ``Observer'' (Spacetime) from the ``System'' (Internal States).

\subsubsection{The Symmetric Decomposition (Space vs. Charge)}
\label{sec:symmetricdecompostion}
The projection must preserve the self-duality property in the subsystems to maintain local conservation. The unique symmetric splitting of $E_8$ is:
\begin{equation}
    E_8 \to D_4 \oplus D_4
\end{equation}

While other decompositions exist (e.g., $E_8 \to A_8$), the $D_4 \oplus D_4$ split is the unique decomposition among maximal rank subgroups that preserves the self-duality of the subspaces \cite{conway_sphere_1988}. By Conway \& Sloane [8] (Ch 4, \S 4.7), other maximal subgroups (e.g., $A_8$, $D_8$) fail to preserve the integer norm condition required for a unitary lattice projection without introducing scaling factors that break self-duality.

This decomposition partitions the 8 dimensions into two orthogonal sectors with distinct physical roles:
\begin{enumerate}
    \item \textbf{Sector A (External Spacetime):} The first $D_4$ lattice defines the coordinate addresses of the lattice nodes. Since $\text{Rank}(D_4)=4$, the observable universe is strictly fixed at \textbf{$D=4$}.
    \item \textbf{Sector B (Internal Symmetry):} The second $D_4$ lattice encodes the internal state (charge, spin, isospin) at each coordinate. These dimensions do not manifest as spatial directions but as the \textbf{Gauge Symmetries} of the Standard Model.
\end{enumerate}
This structural partition explains why the universe appears 4-dimensional while possessing complex internal forces, without requiring the hidden spatial dimensions of Kaluza-Klein theory.

\subsubsection{The Bit-Depth of the Node (\texorpdfstring{$\nu = 16$}{nu16})}
\label{sec:chiralcapacity}

We must determine the information capacity (Bit-Depth) of the lattice nodes. What is the minimal geometric structure required to define a persistent, distinguishable signal on the lattice?

The $E_8 \to D_4 \oplus D_4$ decomposition creates a local geometry governed by $Spin(8)$. However, this structure faces a critical information-theoretic limitation that prevents it from satisfying the \textbf{Identity} and \textbf{Causality} pillars.

\begin{enumerate}
    \item \textbf{The Distinguishability Constraint (Identity):} 
    The spinor representations of $Spin(8)$ are mathematically \textbf{Real} (Self-Conjugate), meaning they lack complex phase information. In signal processing terms, this means a state vector is identical to its conjugate ($\psi = \bar{\psi}$). A system built on this logic cannot distinguish a signal from its inverse (Phase Ambiguity). To support a stable Identity, the system requires \textbf{Complex Representations} ($\psi \neq \bar{\psi}$), allowing for the encoding of phase information distinct from amplitude.
    
    \item \textbf{The Directionality Constraint (Causality):} 
    To support the \textbf{Causality} pillar (Arrow of Time), the system must distinguish ``Input'' from ``Output.'' Geometrically, this requires \textbf{Chirality}, the ability to distinguish Left-handed projections from Right-handed projections. Odd-dimensional rotation groups (like $Spin(9)$) possess complex spinors but lack chiral distinction.
    
    \item \textbf{The Minimal Extension:} 
    We seek the minimal geometric group rank that satisfies both conditions:
    \begin{itemize}
        \item Complex (to encode Phase/Identity)
        \item Chiral (to encode Flow/Causality)
    \end{itemize}

    The smallest group containing $Spin(8)$ that supports both complex and chiral representations is $Spin(10)$. $Spin(9)$ admits complex spinors but lacks chiral decomposition; $Spin(10)$ is the minimal even-dimensional extension.
\end{enumerate}

The size of the fundamental data packet (spinor) in this minimal valid geometry is:
\begin{equation}
    \nu = 2^{\frac{10}{2}-1} = 2^4 = \mathbf{16}
\end{equation}

Thus, $\nu = 16$ is not an arbitrary particle count; it is the minimum sufficient geometric Bit-Depth required to encode complex, directed information on the lattice.

\paragraph{Physical Correlate:} 
This geometry, established strictly to satisfy informational constraints, necessarily manifests as the existence of \textbf{Fermions} (Matter). The Complex requirement ($\psi \neq \bar{\psi}$) creates the distinction between Matter and Antimatter. The Chiral requirement creates the Parity violation observed in the Weak Interaction. These are not inputs to the theory, but inevitable consequences of the system's requirement for directed information processing.

\paragraph{The Total State Space ($N$)}
\label{sec:spacen}
The projection creates two orthogonal sectors (Matter and Mirror) to preserve self-duality. Consequently, the total information capacity of a single lattice node is the sum of these two chiral halves:
\begin{equation}
    N = \nu_{\text{matter}} + \nu_{\text{mirror}} = 16 + 16 = \mathbf{32}
\end{equation}
This integer $N = 32$ represents the total number of distinct state channels that must be mapped onto the temporal dimension without collision.


\subsection{The Encoding Architecture (Metric Signature)}
\label{sec:metric_signature}

Having established a 4D manifold with $\nu=16$ channel capacity, we must determine how to encode information on this substrate without loss (\textbf{Protocol} pillar) or ambiguity (\textbf{Identity} pillar).

The encoding architecture is determined by the \textbf{metric signature}, the algebraic structure governing how information states combine and transform. We show that the requirements of lossless encoding uniquely determine this signature.

\subsubsection{The Encoding Requirements}
The decomposition $E_8 \to D_4 \oplus D_4$ produces a 16-dimensional real representation. To embed this information into a 4-dimensional spacetime manifold, the algebra must support:
\begin{enumerate}
    \item \textbf{Complex Structure:} The ability to represent 16 real dimensions as 8 complex dimensions ($16\mathbb{R} \to 8\mathbb{C}$). This acts as a geometric compression scheme.
    \item \textbf{Chiral Decomposition:} The ability to split these 8 complex dimensions into two orthogonal 4-dimensional subspaces ($8\mathbb{C} \to 4\mathbb{C}_L + 4\mathbb{C}_R$).
\end{enumerate}

These are purely geometric encoding requirements. We now demonstrate that satisfying them in 4 dimensions uniquely determines the metric signature by analyzing the \textbf{Clifford Algebra} $Cl(p,q)$ and its \textbf{Pseudoscalar} (Volume Element) $\omega = e_1 e_2 e_3 e_4$.

\begin{itemize}
    \item \textbf{Euclidean ($4,0$):} In a space with metric $(+,+,+,+)$, the pseudoscalar squares to positive unity:
    \begin{equation}
        \omega^2 = e_1 e_2 e_3 e_4 e_1 e_2 e_3 e_4 = +1
    \end{equation}
    Because $\omega^2 \neq -1$, the algebra $Cl(4,0)$ lacks a geometric imaginary unit. It is isomorphic to $M_2(\mathbb{H})$ (Quaternions). Quaternionic spinors are symplectic (Real), meaning the encoding cannot support the complex compression required for $\nu=16$. Furthermore, the chiral projectors $P_{\pm} = \frac{1}{2}(1 \pm \omega)$ are Real, preventing the phase separation required for distinct Left/Right states.
    
    \item \textbf{Lorentzian ($3,1$):} In a space with metric $(-,+,+,+)$, the inclusion of a timelike basis vector $e_0$ (where $e_0^2 = -1$) alters the pseudoscalar properties. The volume element $\omega = e_0 e_1 e_2 e_3$ squares to negative unity:
    \begin{equation}
        \omega^2 = (e_0 e_1 e_2 e_3)(e_0 e_1 e_2 e_3) = -1
    \end{equation}
    This allows $\omega$ to function as the \textbf{Geometric Imaginary Unit} ($i \equiv \omega$). This naturally generates the complex structure $\mathbb{C}$ required to compress the 16 real states into 8 complex states. Consequently, the chiral projectors $P_{\pm} = \frac{1}{2}(1 \pm i)$ become orthogonal complex operators, enabling the distinct encoding of Left and Right chiral states ($\nu_L + \nu_R$).
\end{itemize}

\paragraph{Physical Consequence (Matter and Antimatter)}
Once these algebraic structures are established to satisfy the encoding requirement, they manifest in physics as fundamental properties of matter. The \textbf{Matter/Antimatter} distinction emerges from the complex conjugation ($\psi \leftrightarrow \bar{\psi}$) enabled by the imaginary unit. \textbf{Chirality} emerges from the orthogonal projectors. These are not postulates, but unavoidable consequences of encoding a 16-dimensional lattice state onto a 4-dimensional Lorentzian manifold.

\paragraph{Note on Metric Emergence (The Cost of Processing)}
This derivation clarifies that "Time" ($dt^2 < 0$) is not a spatial dimension. It is the \textbf{Algebraic Generator} of the complex unit $i$.
To maintain Unitarity (Information Conservation), the total information content of an interval must be invariant. Because Time represents the \textit{consumption} of calculation capacity (processing), while Space represents the \textit{allocation} of capacity (storage), they act as conjugate variables with opposite metric signs. The invariant interval $ds^2 = -dt^2 + dx^2$ is the conservation law of the processing substrate.


\subsubsection{The Chiral Diode (The Arrow of Time)}
\label{sec:chiral_diode}

To enforce causal order, the lattice must distinguish ``Past'' from ``Future.'' The $E_8$ root system possesses a total capacity of $N=32$ degrees of freedom, decomposing symmetrically as:
\begin{equation}
    E_8 \to D_4 \oplus D_4 \quad \Rightarrow \quad N = \nu_L + \nu_R = 16 + 16 = 32
\end{equation}
However, a fully symmetric channel allows information to flow bidirectionally (standing waves), which prohibits the formation of a temporal gradient.

\textbf{The Mechanism:} The truncation is the geometric consequence of the \textbf{Signature Change} from Euclidean ($8,0$) to Lorentzian ($3,1$) geometry. The introduction of the timelike metric signature ($-+++$) mathematically necessitates the splitting of the 32-component real spinor into two complex Weyl spinors ($16_L + 16_R$).

To preserve norm positivity along the negative-signature temporal axis, the projection must couple exclusively to one chiral sector. This acts as a \textbf{Geometric Diode}: only signals propagating ``with the grain'' of the chiral projector are permitted to drive state updates, thereby enforcing the Arrow of Time.


\subsection{The System Logic (\texorpdfstring{$\sigma$=5}{sigma=5} and \texorpdfstring{$\chi$=2}{chi=2})}
\label{sec:derivation_d}
\label{sec:system4_derivationd_sytem_logic}
We derive the rank of the interaction symmetry ($\sigma$). This derivation is supported by two converging lines of evidence: one from Group Theory (Algebraic) and one from Manifold Geometry (Geometric).

\subsubsection{The Topological Boundary (\texorpdfstring{$\chi=2$}{chi=2})}
For a particle to be distinct from the vacuum, its boundary must strictly separate the universe into two disjoint sets: ``Inside'' (The System) and ``Outside'' (The Environment).  We assert that the topological boundary of a persistent particle must be a sphere ($\chi=2$) as the \emph{unique} solution to the \textbf{Binary Partition Constraint}.

We derive this by analyzing the Euler Characteristic for closed, orientable surfaces:
\begin{equation}
    \chi = 2 - 2g
\end{equation}
where $g$ is the genus (number of holes).

\begin{itemize}
    \item \textbf{The Exclusion of $g \geq 1$ (The Leaky Partition):} Any topology with one or more holes (Torus $g=1$, Double Torus $g=2$, etc.) fails to define a strict binary separation.
    \begin{itemize}
        \item \textit{Information-Theoretic Failure:} A genus $g \geq 1$ surface is not simply connected. It supports non-contractible loops—paths that thread through the holes without intersecting the surface. This creates informational ambiguity: field lines can interact with the topology without being ``enclosed,'' rendering the definition of total charge ambiguous (Gauss's Law fails).
        
        \item \textit{Thermodynamic Failure:} By the Gauss-Bonnet theorem ($\int_M K \, dA = 2\pi\chi$), any surface with $g \geq 1$ has $\chi \leq 0$, implying neutral or negative total curvature. Such a surface cannot support net positive internal pressure (energy density) against the vacuum, it would structurally collapse.
    \end{itemize}
    
    \item \textbf{The Uniqueness of $g=0$ (The Sphere):} The sphere is the unique closed surface with $g=0$, yielding $\chi=2$, the maximum possible value. It is the only simply connected topology, ensuring that all loops contract to a point. This forces every field line to be explicitly either contained or excluded, enabling the perfect binary partition of state required for a persistent, distinguishable particle.
\end{itemize}

\paragraph{Physical Consequence: Charge Quantization}
The invariant $\chi=2$ is the \emph{necessary and sufficient} condition for \textbf{charge quantization}. Because $\chi$ must be an integer and $\chi=2$ is the unique maximum for closed surfaces, the charge associated with this topology is discrete. You can have 1 sphere or 2 spheres, but not 1.5 spheres. This geometric constraint allows the continuous lattice field to support discrete, countable units of charge.

\subsubsection{The Interaction Symmetry (\texorpdfstring{$\sigma=5$}{sigma=5})}
\label{sec:sigma}
\begin{enumerate}
    \item \textbf{Geometric Necessity (Degrees of Freedom):}
    While the minimal Grand Unified Theory group $SU(5)$ specifies $\sigma=5$ algebraically
    \footnote{Algebraic Confirmation (The Container): We observe that this geometric derivation ($\sigma=5$) perfectly predicts the rank of the minimal Grand Unified Theory group, $SU(5)$ \cite{georgi_unity_1974}, which unifies the Standard Model forces. In our framework, this is not a coincidence but a consequence: the gauge group is simply the symmetry group of the geometric degrees of freedom.}, we derive this value from pure geometry:

    A persistent particle must be defined by two orthogonal property sets: \textbf{Location} (Where it is) and \textbf{Boundary} (What it is).
    \begin{itemize}
        \item \textbf{Spatial Freedom ($D_{space} = 3$):} The minimal dimensions required to define translation in a causal manifold ($D-1$).
        \item \textbf{Topological Freedom ($\chi = 2$):} The minimal dimensions required to define a closed, simply-connected boundary surface (The Sphere).

    \end{itemize}
    Since translation operations commute with boundary deformations, these sectors are orthogonal. Thus, the minimal embedding vector space $V$ must span their sum:
    \begin{equation}
        \sigma = \dim(V) = 3 \text{ (Space)} + 2 \text{ (Boundary)} = \mathbf{5}
    \end{equation}
\end{enumerate}
This dual convergence identifies $\sigma=5$ as the inevitable Interaction Order.


\subsection{The Fundamental Resonance (\texorpdfstring{$\Delta=43$}{Delta=43})}
\label{sec:fundamental_resonance}
\label{sec:derivation_e}
Finally, we derive the fundamental frequency of the lattice. This is the only dynamic integer in the set. It must satisfy three simultaneous filters to support a persistent universe.

\subsubsection{Filter 1: Unitarity and The Uniqueness Constraint}
\label{sec:fundamental_resonance_filter1}
For information to be conserved (Unitarity), the evolution of a state must be unambiguous and perfectly reversible. We translate this physical requirement into a mathematical one by postulating that the algebraic structure governing the lattice dynamics must be a \textbf{Unique Factorization Domain (UFD)}.

The reason is information-theoretic: in a discrete causal set, the \textit{history} of a state is defined by the sequence of algebraic operations (factors) that generated it. If the algebraic field has a class number $h > 1$, the Fundamental Theorem of Arithmetic fails; a single state norm can be decomposed into multiple non-equivalent sets of prime factors.

This creates \textbf{Causal Ambiguity}: the system cannot uniquely reconstruct its past from its current state. This violates the Unitary requirement of information conservation ($ \psi^{\dagger}\psi = 1 $). Therefore, for a universe to preserve its own history (Unitary), the underlying lattice algebra must be a Unique Factorization Domain (UFD), strictly requiring $h=1$.

The Stark-Heegner theorem \cite{stark_complete_1967} is a mathematical classification result, not a physical 
postulate. It provides the complete, finite list of values for $\Delta$ that satisfy the requirement of $h=1$:
\begin{equation}
    \Delta \in \{1, 2, 3, 7, 11, 19, 43, 67, 163\}
\end{equation}
These nine \textbf{Heegner numbers} are the only candidates for the fundamental resonance of a unitary universe.

\subsubsection{Filter 2: Causality (The ``Bandwidth'' Constraint)}
\label{sec:fundamental_resonance_filter2}
To apply the causality constraint, we must first determine the total information capacity ($N$) of the $E_8$ lattice substrate. The symmetric decomposition of the lattice, $E_8 \rightarrow D_4 \oplus D_4$, creates two orthogonal 4-dimensional sectors. Each sector supports a chiral spinor representation of dimension $\nu = 16$. The total state space of a single lattice node is therefore the sum of these two chiral halves:
\begin{equation}
    N = \nu_{\text{matter}} + \nu_{\text{mirror}} = 16 + 16 = 32
\end{equation}
This integer $N=32$ represents the total number of distinct state channels that must be mapped onto the temporal dimension without collision. With this capacity now defined, we can apply the causal filter.

\begin{itemize}
    \item \textbf{The Problem:} The system must map these $N=32$ parallel channels onto the serial timeline defined by $\Delta$. If the timeline cycle ($\Delta$) is shorter than the number of channels ($N$), the \textbf{Pigeonhole Principle} forces at least two distinct states to map to the same time coordinate. This creates ``Causal Aliasing'' (a signal collision), which destroys the history of the system.

    \item \textbf{The Constraint:}: To ensure every state has a unique temporal address and to maintain a non-zero persistence margin against fluctuations, the projection must satisfy the strict inequality $\Delta > N$. Therefore, we require $\Delta > 32$.
    
    \item \textbf{Remaining Candidates:} Applying this constraint to the set of Heegner numbers leaves: $\{43, 67, 163\}$.
\end{itemize}

\subsubsection{Filter 3: Temporal Atomicity (The Coordination Constraint)}
\label{sec:fundamental_resonance_filter3}

The temporal modulus $\Delta$ is defined by Filter 1 as the \textbf{fundamental, indivisible clock period}. This atomicity requirement imposes a strict upper bound on the cycle length.

While the physical consequence of this bound is the stabilization of the \textbf{Spinor Double Cover}\footnote{
    \textbf{Physical Anticipation:} In the derived physics (System 1), fermions live on the spinor double cover of the manifold, requiring a $4\pi$ rotation ($\psi \to -\psi \to \psi$) to return to identity. This implies the effective information content is doubled ($2N$).
    For the temporal coordinate to resolve this topology without ambiguity, the lattice update cycle $\Delta$ must maintain \textbf{Phase Lock} with the signal. If the non-signaling guard interval ($M = \Delta - N$) exceeds the signal length ($N$), the phase history is lost in the gap. 
    Mathematically, strict causality requires $\Delta < 2N$. If $\Delta \ge 2N$, the vacuum gap exceeds the particle coherence length, creating \textbf{Topological Aliasing} where the phase winding number becomes undefined, destroying fermionic statistics.
}, System 1 must depend only the principles of Information Energetics to derive it. We do not assume fermions exist yet; we assume only a system optimizing for persistence.

\begin{itemize}
    \item \textbf{The Mechanism (Signal vs. Overhead):} 
    The fundamental cycle $\Delta$ partitions into $N=32$ active signal states (the information payload) and $M = \Delta - N$ overhead intervals (the processing latency or guard bands).
    
    \item \textbf{The Constraint (Causal Continuity):} 
    Because $\Delta$ is atomic (possessing no faster internal clock), the $M$ overhead intervals cannot be self-regulating; they must be causally bridged by the active signal states. If an overhead interval exists without a corresponding signal controller, the system experiences an \textbf{Acausal Gap}—a period of time defined by the metric but undefined by the state.
    
    \item \textbf{The Proof (Surjective Mapping):} 
    To maintain causal continuity, every overhead interval must be mapped to at least one controlling signal state. Mathematically, let $S$ be the set of signal states ($|S|=N$) and $O$ be the set of overhead intervals ($|O|=M$). The system requires a \textbf{Surjective Function} $f: S \twoheadrightarrow O$.
    
    By the definition of surjection, the domain must be greater than or equal to the codomain ($|S| \ge |O|$).
    \begin{equation}
        N \ge M \implies N \ge (\Delta - N) \implies \Delta \le 2N
    \end{equation}
    With $N=32$, the hard upper limit for the atomic update cycle is $\Delta \le 64$.
\end{itemize}

\textbf{Result:} $\Delta = 43$ is the unique solution satisfying Unitarity (Filter 1), Causality (Filter 2), and Temporal Atomicity (Filter 3).

\subsection{The System Specification}
\label{sec:system1spec}
The $E_8$-Persistence theory fulfills the six pillars of persistence ontop of a substrate.

\begin{itemize}
    \item \textbf{Substrate: Manifold Rank} ($D=4$)
    \item \textbf{Capacity ($\Delta E$): Fundamental Resonance ($\Delta=43$).} 
    The maximum non-repeating frequency of the lattice (Heegner Number), representing the bit-depth of the vacuum.
    \item \textbf{Identity ($\Delta I$): Interaction Symmetry ($\sigma=5$).} 
    The geometric rank required to encode the unified force ($SU(5)$ precursor).
    \item \textbf{Protocol ($MI$): Chiral Capacity ($\nu=16$).} 
    The active degrees of freedom available for matter storage (The Weyl Spinor).
    \item \textbf{Governor ($G$): Topological Boundary ($\chi=2$).} 
    The Euler characteristic required for closed loops (stable particles), enforcing the finiteness of the field.
    \item \textbf{Temporal Cost ($T$): Causality ($-1$).} 
    The metric signature requirement for state updates, creating the arrow of time. 
    \item \textbf{Persistence Margin ($PM$): ($+3$)} 
    The dimensional embedding required to support the projection.   
\end{itemize}

\subsection{Synthesis: The Characteristic Integers}

We have now completed the derivation of the vacuum's fundamental hardware. The constraints of Finiteness, Unitarity, and Causality have led us not to a family of possibilities, but to a singular mathematical object defined by a unique set of five \textbf{Characteristic Integers}.

\begin{equation}
    \mathbb{S} = \{D=4, \Delta=43, \nu=16, \sigma=5, \chi=2\}
\end{equation}

This set represents the complete and immutable specification of System I. These are not free parameters or empirical inputs; they are the derived architectural constants of a persistent reality. \Cref{sec:geometric_impedance} will use this set as its sole input to perform the theory's first physical calculation.