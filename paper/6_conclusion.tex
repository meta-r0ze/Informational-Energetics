\section{Conclusion}

This work introduced \textit{Informational Energetics}, a framework to describe any system that persists. The physics test was designed as the ultimate crucible for IE. We proposed that the fundamental laws and structures of reality are not arbitrary axioms, but are the emergent and necessary consequences of a universe optimized for persistence.

This was not a search over possibilities, but a sequence of constraints that eliminates all but a single, unique solution for the substrate of reality. The $E_8$ lattice, projected onto a causal 4D manifold, yields four immutable integers—the Characteristic Integers $\mathbb{S} = \{ \Delta=43, \nu=16, \sigma=5, \chi=2 \}$ derived with no free parameters. This creates the foundation of the $E_8$-Persistence theory.

The ultimate test of any such ontological framework is its ability to make falsifiable contact with phenomenology. We meet this requirement by deriving a parameter-free calculation for the Fine-Structure Constant that agrees with the CODATA value to within $1.68\sigma$. This result simultaneously fixes the values of the Von Klitzing Constant and the Elementary Charge. This shows that the constants of nature are coupled manifestations of the vacuum's geometric impedance; specifically, the ratio of this impedance to the \textit{Margin} defines the Bohr Radius ($a_0 \propto Z_{geo}/Z_{MAR}$), structurally locking the scale of the atom to the resolution limit of the vacuum.

As detailed in subsequent papers this geometric framework resolves the Standard Model parameter inventory, including the gauge couplings, fermion masses, and mixing hierarchies, without introducing free parameters. The framework contains no adjustable parameters: each prediction either matches experiment or falsifies the theory entirely.

This is not merely the continuation of a physical theory, but the beginning of a scientific paradigm that treats the laws governing physics, biology, computation, and other persistent systems not as disparate subjects, but as specific instantiations of a single, universal architecture of persistence.

\begin{acknowledgments}
The author is an independent researcher and received no external funding for this work. 

I would like to thank Brian Sheppard for rigorous and constructive feedback.

I would like to thank my friends and family for their patience and support throughout decades of discussions as I tried to understand every field I became interested in and the recurring, overarching patterns I saw.
\end{acknowledgments}