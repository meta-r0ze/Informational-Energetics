\begin{abstract}
We present Informational Energetics (IE), a unified framework for modeling the persistence of complex adaptive systems against entropy. By synthesizing algorithmic information theory, 
non-equilibrium thermodynamics, and control theory, we derive the Universal Architecture of Persistence: a recursive, six-dimensional set of constraints (Capacity, Identity, Protocol, Governor, Temporal Cost, and Persistence Margin) that any system must satisfy to maintain structural coherence.

To validate the framework's universality, we apply IE to the most fundamental persistent system: the vacuum. This paper establishes the theoretical foundation for the $E_8$-Persistence Theory demonstrating that spacetime and the Standard Model are not arbitrary structures but the unique solutions to three fundamental requirements: Finiteness (preventing divergence), Unitarity (conserving information), and Causality (enforcing temporal order). The theory makes a falsifiable prediction: all physical constants derive from five geometric integers $\mathbb{S} = \{D{=}4, \Delta{=}43, \nu{=}16, \sigma{=}5, \chi{=}2\}$ with zero free parameters. We derive these values purely from persistence requirements, without empirical input. Any requirement for parameter tuning would falsify not merely a physical theory, but the fundamental claim that persistence itself has an invariant architecture.

% alpha^-1
\CatchFileBetweenTags{\AlphaInvVal}{calculations/constants.tex}{AlphaInvVal}
\CatchFileBetweenTags{\AlphaInvExperimentalValue}{calculations/constants.tex}{AlphaInvExperimentalValue}
We validate this requirement by calculating the persistent Geometric Impedance of the vacuum (fine-structure constant) as the unique sum of the canonical costs defined by these five integers.
\begin{equation}
\alpha^{-1} = \underbrace{\pi\Delta}_{\Delta E}
+ \,\underbrace{\chi}_{\Delta I}
- \,\underbrace{\frac{1}{D\Delta - \sigma}}_{MI}
- \,\underbrace{\frac{\chi}{\Delta}}_{G}
+ \,\underbrace{\frac{1}{N^3} \cdot \frac{\chi}{\sigma} \cdot \left( 1 - \frac{\sigma}{D\Delta} \right)}_{T}
+ \,\underbrace{\frac{1}{H_{full} \cdot (\sigma + 1) \cdot \Delta^2}}_{PM}
\end{equation}
Our parameter-free calculation yields $\alpha^{-1} = \AlphaInvVal \dots$, a value in agreement with the CODATA 2022 consensus to within 1.68$\sigma$ and within $0.58\sigma$ of the Morel 2020 value.

\end{abstract}