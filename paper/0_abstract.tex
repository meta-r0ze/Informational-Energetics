\begin{abstract}
We present Informational Energetics (IE), a unified framework for modeling the persistence of complex adaptive systems against entropy. By synthesizing algorithmic information theory, non-equilibrium thermodynamics, and control theory, we derive the Universal Architecture of Persistence: a recursive, six-dimensional set of constraints (Capacity, Map, Protocol, Governor, Toll, and  Margin) that any system at any scale must satisfy to maintain structural coherence.

To validate the framework's universality, we apply IE to the most fundamental persistent system: the vacuum. This paper establishes the theoretical foundation for the $E_8$-Persistence Theory demonstrating that spacetime and the Standard Model are not arbitrary structures but the unique solutions to fundamental requirements: Finiteness (preventing divergence), Unitarity (conserving information), and Causality (enforcing temporal order). We show, without using any empirical input, that an $E_8$ lattice projected onto spacetime is the unique solution to these persistence requirements. The theory makes a falsifiable prediction: all physical constants derive from this geometry, whose invariants are the integers $\mathbb{S} = \{D{=}4, \Delta{=}43, \nu{=}16, \sigma{=}5, \chi{=}2\}$ with zero free parameters. Any requirement for parameter tuning would falsify not merely a physical theory, but the fundamental claim that persistence itself has an invariant architecture.

% alpha^-1
\CatchFileBetweenTags{\AlphaInvVal}{calculations/constants.tex}{AlphaInvVal}
\CatchFileBetweenTags{\AlphaInvExperimentalValue}{calculations/constants.tex}{AlphaInvExperimentalValue}
We validate this prediction by calculating the persistent Geometric Impedance of the vacuum (fine-structure constant) as the unique sum of the canonical costs.
\begin{equation}
\alpha^{-1} = \underbrace{\pi\Delta}_{CAP}
+ \,\underbrace{\chi}_{MAP}
- \,\underbrace{\frac{1}{D\Delta - \sigma}}_{PRO}
- \,\underbrace{\frac{\chi}{\Delta}}_{GOV}
+ \,\underbrace{\frac{1}{N^3} \cdot \frac{\chi}{\sigma} \cdot \left( 1 - \frac{\sigma}{D\Delta} \right)}_{TOL}
+ \,\underbrace{\frac{1}{H_{full} \cdot (\sigma + 1) \cdot \Delta^2}}_{MAR}
\end{equation}

Our parameter-free calculation yields $\alpha^{-1} = \AlphaInvVal \dots$, a value in agreement with the CODATA 2022 consensus to within 1.68$\sigma$ and within $0.58\sigma$ of the Morel 2020 value.
\end{abstract}