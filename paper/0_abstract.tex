\begin{abstract}
We present Informational Energetics (IE), a unified framework for modeling the persistence of complex adaptive systems against entropy. By synthesizing information theory, non-equilibrium thermodynamics, and control theory, we derive the Universal Architecture of Persistence: a recursive, six-dimensional set of constraints (Capacity, Map, Protocol, Governor, Toll, and  Margin) that any system at any scale must satisfy to maintain structural coherence.

To validate the framework's universality, we apply IE to the most fundamental persistent system: the vacuum. This paper establishes the theoretical foundation for the $E_8$-Persistence Theory by demonstrating that spacetime and the Standard Model are the unique solutions to fundamental requirements: Finiteness (preventing divergence), Unitarity (conserving information), and Causality (enforcing temporal order). We show that the unique solution is an $E_8$ lattice projected onto $4D$ spacetime whose invariants are the integers $\mathbb{S} = \{\Delta{=}43, \nu{=}16, \sigma{=}5, \chi{=}2\}$.

\CatchFileBetweenTags{\AlphaInvVal}{calculations/constants.tex}{AlphaInvVal}
\CatchFileBetweenTags{\AlphaInvExperimentalValue}{calculations/constants.tex}{AlphaInvExperimentalValue}
The theory makes a falsifiable prediction: all physical constants derive from this geometry with zero free parameters. We validate this prediction by defining the Geometric Impedance. This yields the value of the Fine-Structure Constant ($\alpha^{-1}$) and structurally fixes the Von Klitzing Constant ($R_K$) and the Elementary Charge ($e$) as necessary consequences.
\begin{equation}
\alpha^{-1} = \underbrace{\pi\Delta}_{\text{Capacity}}
+ \,\underbrace{\chi}_{\text{Map}}
- \,\underbrace{\frac{1}{D\Delta - \sigma}}_{\text{Protocol}}
- \,\underbrace{\frac{\chi}{\Delta}}_{\text{Governor}}
+ \,\underbrace{\frac{1}{N^{D_{space}}} \cdot \frac{\chi}{\sigma} \cdot \left( 1 - \frac{\sigma}{D\Delta} \right)}_{\text{Toll}}
+ \,\underbrace{\frac{1}{L_{embed} \cdot (\sigma + 1) \cdot \Delta^2}}_{\text{Margin}}
\end{equation}
Our parameter-free calculation yields $\alpha^{-1} = \AlphaInvVal \dots$, a value in agreement with the CODATA 2022 consensus to within 1.68$\sigma$ and within $0.58\sigma$ of the Morel 2020 value.
\end{abstract}