\section{Introduction}

The convergence of complex systems theory, non-equilibrium thermodynamics, and algorithmic information theory suggests that the structures we observe in nature, from biological cells to economic markets, are not arbitrary. They are the surviving solutions to a universal optimization problem: how to maintain order in an entropic environment. Although biology and economics explicitly attempt to model this struggle for persistence, fundamental physics has traditionally treated the structure of reality as a static background. Spacetime dimensions, gauge symmetries, and coupling constants are accepted as axiomatic inputs, measured by experiment but unexplained by theory.

We propose that all persistent systems including physics are the emergent limits of a deeper, substrate-independent logic governing how information must be processed to resist thermodynamic equilibrium.

\subsection{The Proposal: Informational Energetics}
Informational Energetics (IE) is the study of how systems convert energy into structural information to delay the onset of thermodynamic equilibrium (death). We synthesize principles from several disciplines including Information theory, Landauer's Principle (information is physical), Thermodynamics, and Control Theory (stability requires feedback) to derive the \textbf{Universal Architecture of Persistence}.

We assert that any persistent entity must possess a specific six-dimensional set of constraints: Capacity, Identity, Protocol, Governor, Temporal Cost, and Persistence Margin. These six ``Pillars of Persistence'' are derived and explained in detail in \cref{sec:IE}.

\subsection{Testing the Framework}

To test the predictive power of Informational Energetics, we apply it to the most constrained system available: the quantum vacuum. Unlike biological or economic systems, the vacuum has zero free parameters and is measured to 12 significant figures. If IE correctly predicts the Standard Model constants, this validates the framework's universality. We utilize the Standard Model not as a paradigm to be replaced, but as a blind test. 

\subsubsection{The Standard Model Ansatz}

In standard formulations, the dimensionality of spacetime ($D=4$) and the geometric invariants of the gauge groups are treated as a pre-existing stage upon which quantum fields act. Standard theory offers no structural reason why time exists as a distinct metric signature necessary for causality, or why the fine-structure constant has a specific value required for stable matter.

We reverse this order of inquiry. We do not start with observation; we start with the \textbf{Requirements of Existence}.

We map IE to physics, treating the vacuum not as a void but as a persistent system. ``System 0'' provides the specific architectural constraints (Finiteness, Unitarity, Causality) of maintaining structural coherence against entropy. We then treat the derivation of physics as an engineering problem: what mathematical structure is capable of satisfying these requirements simultaneously?

We demonstrate that the geometric invariants of the Standard Model are not arbitrary. They are the unique mathematical solutions to the impedance matching problem defined by the IE architecture. We show that:
\begin{itemize}
    \item The requirement of \textbf{Finiteness} and \textbf{Homogeneity} necessitates a Lattice Substrate.
    \item The requirement of \textbf{Unitarity} selects the $E_8$ lattice (via Kneser's Theorem).
    \item The requirement of \textbf{Causality} enforces a geometric projection from Euclidean space to Lorentzian spacetime ($E_8 \to D_4 \oplus D_4$).
\end{itemize}

% alpha^-1
We then derive the fine-structure constant as the geometric impedance—the entropic cost of sustaining 
a topological charge against the $E_8$ lattice flux matching experimental values and to eliminate the 
possibility of reverse-engineering, we perform a Monte Carlo diffusion audit on the $E_8$ lattice with zero input parameters. The simulation independently produces $\alpha^{-1} = 136.979 \pm 0.105$, 
confirming that the fine-structure constant emerges from pure lattice topology.

This derivation provides a direct, falsifiable test of the theory: if existence implies specific constraints, the constants of nature must match the unique solutions of those constraints.

\subsection{Theoretical Context}

\subsubsection{The Information-Theoretic Turn}
The concept that physical reality is fundamentally information processing is rooted in the work of Wheeler (``It from Bit'') \cite{wheeler_information_1989} and Landauer \cite{landauer_irreversibility_1961}. More recently, Verlinde proposed that gravity is an entropic phenomenon emerging from information gradients \cite{verlinde_origin_2011}.

While concordant with Verlinde and Landauer, IE applies this logic broadly to all persistent systems, treating the minimization of Entropic Action as the primary driver of lattice dynamics.

The following section formalizes this information-theoretic approach as \textbf{Informational Energetics}, establishing the universal structural requirements that any persistent system, including the vacuum, must satisfy. All subsequent derivations follow from applying these principles including deriving the mathematical structure of the $E_8$ lattice and the branching rules catalogued by Slansky~\cite{slansky_group_1981}.

\subsubsection{The \texorpdfstring{$E_8$}{E8} Lattice: Substrate vs. Algebra}

The exceptional Lie group $E_8$ has long been explored as a candidate for unification due to its status as the largest finite simple symmetry group. Most famously, Lisi proposed embedding the Standard Model directly into the $E_8$ algebra \cite{lisi_exceptionally_2007}. However, Distler and Garibaldi demonstrated that a direct algebraic embedding cannot reproduce the chiral structure of the Standard Model without introducing mirror fermions that are not observed \cite{distler_there_2010}.

We explicitly depart from the algebraic embedding approach. We treat $E_8$ not as the Gauge Algebra (the effective field), but as the \textbf{Geometric Substrate} (the fundamental hardware). By applying Kneser's Theorem \cite{kneser_klassenzahlen_1957}, we derive physics from the \textit{projection} of the $E_8$ lattice onto a 4-dimensional manifold ($E_8 \to D_4 \oplus D_4$).

In this framework, chirality emerges strictly from the \textbf{geometric projection} involving a metric signature change, rather than algebraic embedding. As detailed in \cref{app:geometric_origins_of_chiral_fermions}, this projection places the model outside the assumptions of the Distler-Garibaldi theorem, allowing for the emergence of chiral fermions without mirror partners.

\subsection{Distinction From Physics-First Approaches}
Several research programs explore discrete spacetime structures (Causal Set Theory \cite{bombelli_space-time_1987}, Loop Quantum Gravity \cite{rovelli_quantum_2004}) or emergent gravity (Verlinde \cite{verlinde_origin_2011}, Jacobson \cite{jacobson_thermodynamics_1995}). These are \emph{physics-first} approaches: they begin with spacetime or gravity and work forward.

Our approach is \emph{systems-first}: we derive universal persistence requirements from information theory and control theory, then show that the vacuum, treated as a persistent system must satisfy them. The Standard Model emerges as a consequence, not a starting assumption. This yields predictions ($\alpha^{-1}$, mixing angles, etc) that physics-first approaches do not address, while simultaneously explaining \emph{why} the SM has the structure it does (gauge groups, generation count, chirality, etc.).