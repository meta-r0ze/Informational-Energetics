\section{Introduction}

The Crisis of Arbitrariness
The Problem: Current science is siloed. Biologists, economists, and physicists study the same patterns (growth, stasis, collapse) but use different math. Physics, specifically, suffers from ``parameter bloat", it describes how fields interact but cannot explain why the constants (masses, couplings) have their specific values.

The Proposal: We reverse the order of inquiry. We do not start with observation; we start with the Requirements of Existence.

The Definition: Informational Energetics (IE) is the study of how systems convert energy into structural information to delay the onset of thermodynamic equilibrium (Death).

\subsubsection{The Information-Theoretic Turn}
The concept that physical reality is fundamentally information processing is rooted in the work of Wheeler (``It from Bit'') \cite{wheeler_information_1989} and Landauer \cite{landauer_irreversibility_1961}. More recently, Verlinde proposed that gravity is an entropic phenomenon emerging from information gradients \cite{verlinde_origin_2011}.

While concordant with Verlinde and Landauer, IE applies this logic broadly to all persistent systems, treating the minimization of Entropic Action as the primary driver of lattice dynamics, and the Selection Principle as the Topological Constraint.

The following section formalizes this information-theoretic approach as \textbf{Informational Energetics}, establishing the universal structural requirements that any persistent system, including the vacuum, must satisfy. All subsequent derivations follow from applying these principles include deriving the mathematical structure of the $E_8$ lattice and the branching rules catalogued by Slansky~\cite{slansky_group_1981}.