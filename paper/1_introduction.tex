\section{Introduction}
The structures we observe: biological cells, economic markets, the quantum vacuum all share a property that physics lacks vocabulary to describe: they \emph{persist}. They maintain identity against the thermodynamic current that erases distinction. 

Standard physics treats the vacuum as a \emph{given}: spacetime dimensions, gauge symmetries, and coupling constants are axiomatic inputs, measured but unexplained. We propose a reversal: the vacuum is not given but \emph{achieved}. It is the minimal solution to a persistence problem: how to maintain 
structural coherence with zero external energy input.

This requires a new theoretical language. Control theory describes how systems regulate against fluctuations, but treats the regulation target (setpoint) as arbitrary. Information theory quantifies the cost of information processing, but treats the processor as given. Thermodynamics establishes the arrow of time, but treats entropy as a property of \emph{states}, not of \emph{processing}.

We synthesize these into \textbf{Informational Energetics}: the study of how systems convert energy into structural information to resist entropic decay. The synthesis yields the \textbf{Universal Architecture of Persistence} six constraints (Capacity, Map, Protocol, Governor, Toll, Margin) that are not merely functional but \emph{existential}: remove any one, and the system's lifetime goes to zero. While this paper validates IE through physics, the framework is constructed to apply to any persistent system.

\subsection{Testing the Framework: The Vacuum as Crucible}

The vacuum is the ideal test case: zero free parameters, maximally precise measurements, and crucially no external environment. It persists against only its own structural possibility. This makes it the purest test of IE: only the internal architecture of persistence itself.

The test proceeds in three stages:
\begin{enumerate}
    \item \textbf{System 0: Requirements.} (\ref{sec:System0_vacuum_specification}) Map IE to physics, deriving Finiteness, Unitarity, and Causality as necessary for vacuum persistence.
    \item \textbf{System 1: Substrate.} (\ref{sec:system1_projection}) Show these requirements uniquely 
    specify an $E_8$ lattice projected to $D=4$ spacetime, yielding the characteristic integers $\mathbb{S} = \{\Delta=43, \nu=16, \sigma=5, \chi=2\}$.
    \item \textbf{System 2: Prediction.} (\ref{sec:system2_geometric_impedance}) The Standard Model has twenty-six free parameters; we derive the first, the fine-structure constant ($\alpha^{-1}$) as a parameter-free geometric consequence of vacuum persistence and validate against measurement.
\end{enumerate}

\subsection{Theoretical Context}

\subsubsection{The Information-Theoretic Turn}
The concept that physical reality is fundamentally information processing is rooted in the work of Wheeler (``It from Bit'') \cite{wheeler_information_1989} and Landauer \cite{landauer_irreversibility_1961}. More recently, Verlinde proposed that gravity is an entropic phenomenon emerging from information gradients \cite{verlinde_origin_2011}.

While concordant with Verlinde and Landauer, IE applies this logic broadly to all persistent systems, treating the minimization of Entropic Action as the primary driver of lattice dynamics.

\Cref{sec:IE} formalizes this information-theoretic approach as \textbf{Informational Energetics}, establishing the universal structural requirements that any persistent system, including the vacuum, must satisfy. All subsequent derivations follow from applying these principles including deriving the mathematical structure of the $E_8$ lattice and the branching rules catalogued by Slansky~\cite{slansky_group_1981}.

\subsubsection{The \texorpdfstring{$E_8$}{E8} Lattice: Substrate vs. Algebra}

The exceptional Lie group $E_8$ has long been explored as a candidate for unification due to its status as the smallest symmetry group large enough to host the Standard Model. Most famously, Lisi proposed embedding the Standard Model directly into the $E_8$ algebra \cite{lisi_exceptionally_2007}. However, Distler and Garibaldi demonstrated that a direct algebraic embedding cannot reproduce the chiral structure of the Standard Model without introducing mirror fermions that are not observed \cite{distler_there_2010}.

We explicitly depart from the algebraic embedding approach. We treat $E_8$ not as the Gauge Algebra (the effective field), but as the \textbf{Geometric Substrate} (the fundamental hardware). In this framework, chirality emerges strictly from the \textbf{geometric projection} involving a metric signature change, rather than algebraic embedding. As detailed in \cref{app:geometric_origins_of_chiral_fermions}, this projection places the model outside the assumptions of the Distler-Garibaldi theorem, allowing for the emergence of chiral fermions without mirror partners.

\subsection{Distinction From Physics-First Approaches}
Several research programs explore discrete spacetime structures (Causal Set Theory \cite{bombelli_space-time_1987}, Loop Quantum Gravity \cite{rovelli_quantum_2004}) or emergent gravity (Verlinde \cite{verlinde_origin_2011}, Jacobson \cite{jacobson_thermodynamics_1995}). These are \emph{physics-first} approaches: they begin with spacetime or gravity and work forward.

Our approach is \emph{systems-first}: we derive universal persistence requirements from information theory and control theory, then determine what substrate must satisfy them. The $E_8$ lattice is not assumed; it is the \emph{unique} geometric object satisfying the requirements. The Standard Model emerges as the necessary consequence of projecting this lattice onto a causal manifold, not as a structure we impose.

This yields predictions ($\alpha^{-1}$, mixing angles, etc) that physics-first approaches do not address, while simultaneously explaining \emph{why} the Standard Model has the structure it does (gauge groups, generation count, chirality, etc.).