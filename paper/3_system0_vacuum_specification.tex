\section{System 0: The Specification of the Vacuum}
\label{sec:System0_vacuum_specification}

The principles of Informational Energetics, if truly universal, must apply to the most fundamental layer of reality: the vacuum itself. We now perform the critical act of translation, applying the universal architecture of persistence to this specific domain. This approach treats the universe not as a given, but as a physical system subject to the same architectural constraints governing information, energy, and stability that are necessary for any entity to persist.

This translation of IE's universal principles into the specific language of physics yields three non-negotiable requirements for the substrate of reality:
\begin{itemize}
    \item \textbf{Finiteness:} ($\Delta E, PM$) To prevent energetic and informational divergence.
    \item \textbf{Unitarity:} ($\Delta I, MI$) To ensure the lossless conservation of information.
    \item \textbf{Causality:} ($G, T$) To enforce a well-defined temporal evolution.
\end{itemize}

The remainder of this section will derive these three properties in detail, building the complete abstract specification for a persistent universe.

\subsection{Persistence Requires \texorpdfstring{$\Delta E, PM$}{EPM} via Finiteness}
For the vacuum to persist, its definition must be self-consistent. A system defined by infinite properties contains no information and cannot maintain a stable structure. In physics, such inconsistencies manifest as energetic divergences, as exemplified by the Ultraviolet (UV) Catastrophe in standard Quantum Field Theory. IE resolves this by enforcing finiteness as a primary requirement of existence. This corresponds to the \textbf{Capacity} and \textbf{Persistence Margin} pillars.

\begin{itemize}
    \item \textbf{The Logic:} Information is physical and requires energy to store \cite{landauer_irreversibility_1961}. A continuous volume, containing infinite potential information, would therefore possess infinite energy and instantly collapse into a singularity.

    \item \textbf{The Requirement:}
    To prevent Energetic Divergence, the substrate must be \textbf{Discrete}: there must exist a fundamental, indivisible unit of information that sets a maximum density.

    To prevent Algorithmic Divergence, the substrate must be \textbf{Homogeneous}. In Algorithmic Information Theory, the structural complexity of a random graph scales with its size ($K(S) \propto N$). For the universe to be scalable without infinite processing overhead, its Kolmogorov Complexity must be $O(1)$, independent of size. This requires that the local topology be the same regardless of location, minimizing the structural description to a single, repeating rule.

    \item \textbf{Specification:} The universe must be a \textbf{Lattice Field}, a structure that is both discrete (finite) and translationally invariant (ordered) rather than a continuous manifold or a random graph.
\end{itemize}

\subsection{Persistence Requires \texorpdfstring{$\Delta I, MI$}{I MI} via Unitarity}
For the vacuum to persist, it must maintain a stable identity over time. This requires that information is perfectly conserved. Since the vacuum \textit{is} the environment, there is no external system to which information can be lost. Therefore, the vacuum must be a perfectly closed and lossless information network. This is the requirement of Unitarity, which maps to the \textbf{Protocol} and \textbf{Identity} pillars.

\begin{itemize}
    \item \textbf{The Requirements:}
    First, to ensure a stable ground state, the \textit{substrate} metric must be \textbf{Positive Definite} (Euclidean).

    \textit{Information-Theoretic Justification}: In a metric space, the norm $|\mathbf{v}|^2 = g_{\mu\nu}v^\mu v^\nu$ represents the information distance from the origin (reference state). For the system to have a well-defined minimum energy configuration, this distance must satisfy:
    \begin{equation}
        |\mathbf{v}|^2 \geq 0 \quad \forall \mathbf{v} \neq 0
    \end{equation}
    A metric with negative eigenvalues (Lorentzian) allows $|\mathbf{v}|^2 < 0$, implying an imaginary ``distance'' from the reference state and preventing the definition of a stable Zero-Point. Furthermore, a Lorentzian metric admits non-trivial null vectors ($|\mathbf{v}|^2 = 0$ where $\mathbf{v} \neq 0$), allowing for the creation of infinite information density without exceeding the capacity budget ($|\mathbf{v}_1|^2 + |\mathbf{v}_2|^2 = 0$). This explicitly violates the \textbf{Finiteness} pillar. Therefore, the substrate metric must be Euclidean.

    \textit{The State Enumeration (The Partition Function):}
    To define the Identity of the system, we must enumerate all accessible configurations. To satisfy \textbf{Finiteness}, the system must be spatially bounded. To satisfy \textbf{Protocol} (Unitarity), it must be closed (no edges). While a sphere satisfies closure, it violates the \textbf{Translational Invariance}: a regular lattice cannot be mapped onto curved geometry without defects. The unique topology that satisfies Finiteness (bounded), Unitarity (closed), and Invariance (flat) is the \textbf{Torus} ($T^n$).
    
    Consequently, the statistical evolution of the vacuum is defined by the Partition Function on a torus:
    \begin{equation}
        Z(\beta) = \sum_{\text{states}} e^{-S_{\text{config}}} = \text{Tr}(e^{-\beta H})\footnote{Here, $\beta$ is the 
formal periodicity parameter of the imaginary time cycle. 
We use the partition function formalism for state 
enumeration on a torus, not to claim the vacuum has a 
literal temperature.}
    \end{equation}
    This function is not merely a thermodynamic tool; it is the fundamental generating function for the system's identity.

    \textit{Requirement for Evenness (Path Independence):}
    A persistent system demands a unique, unambiguous equilibrium state. If the state count $Z$ depended on the path taken through moduli space (parameterization) rather than the configuration itself, the vacuum would lack a stable \textbf{Identity}. 
    
    This requires $Z(\beta)$ to be single-valued under the modular transformation $T: \tau \to \tau+1$. The Jacobi Theta Function, $\Theta_\Lambda(\tau) = \sum e^{i \pi \tau |\mathbf{v}|^2}$, counts these states. For \textbf{Even} lattices ($|\mathbf{v}|^2 = 2n$), the exponent $2\pi i n \tau$ is invariant under the shift. However, any odd-norm vector ($|\mathbf{v}|^2 = 2n+1$) introduces a sign inversion ($\Theta \to -\Theta$), rendering the identity multi-valued. Thus, the lattice must be \textbf{Even}.

    \textit{Requirement for Self-Duality (Read/Write Symmetry):}
    To prevent information loss via destructive interference or Landauer erasure, the encoding operation (write) and the decoding operation (read) must be informationally equivalent. This is enforced by the modular $S$-transformation ($S: \tau \to -1/\tau$), which maps the lattice to its reciprocal (Fourier dual).
    
    By the Poisson Summation formula, the partition function transforms as:
    \begin{equation}
        \Theta_\Lambda(-1/\tau) \propto \frac{1}{\text{vol}(\Lambda)} \Theta_{\Lambda^*}(\tau)
    \end{equation}
    For the Identity to remain invariant ($Z_{\Lambda} = Z_{\Lambda^*}$), the lattice must be \textbf{Self-Dual} ($\Lambda = \Lambda^*$). This strictly enforces \textbf{Unimodularity} ($\text{vol}(\Lambda) = 1$), as the only scalar satisfying $V = 1/V$ is $1$. Any other volume introduces an irreversible scaling factor, violating Unitarity.

    \item \textbf{Specification:} The lattice of the vacuum must be \textbf{Positive Definite, Unimodular, Even, and Self-Dual}.
\end{itemize}

\subsection{Persistence Requires \texorpdfstring{$G, T$}{GT} via Causality}
For the vacuum to persist, it must not only exist stably but also \textit{evolve} in a well-defined manner. This creates a fundamental information-theoretic challenge: how to project the vast state space of a lattice node onto a single, linear temporal axis without ambiguity. This maps to the \textbf{Governor} and \textbf{Temporal Cost} pillars.

\begin{itemize}
    \item \textbf{The Logic:}
    From the \textbf{Capacity} pillar, we established that any persistent system must have a finite maximum entropy. Let $N$ denote this structural limit, the \textbf{Shannon Channel Capacity} of a single lattice site.
    
    To evolve, the system must map these $N$ parallel state options onto a serial timeline defined by a fundamental integer cycle, the \textbf{Temporal Modulus} ($\Delta$). Because the substrate is discrete, $\Delta$ represents the number of discrete update steps in one fundamental clock cycle.

    \item \textbf{The Requirement:}
    By the \textbf{Pigeonhole Principle}, if the timeline's cycle length ($\Delta$) is shorter than the number of state channels ($N$), at least two distinct states must map to the same temporal index. This creates \textbf{Causal Aliasing}, a collision where distinct information states become indistinguishable in time, effectively destroying the system's history.

    To enforce a strict arrow of time, the system must implement a \textbf{Geometric Governor} that prohibits aliasing. The condition $\Delta = N$ represents a critically saturated channel. To persist \textit{robustly} against fluctuations, the temporal container must be strictly larger than the information content it holds.

    \item \textbf{Specification:}
    The projection must satisfy the strict inequality $\Delta > N$. The vacuum is defined by a \textbf{Causal Projection} where the update cycle length exceeds the node capacity.
\end{itemize}

\subsection{Summary: The Architectural Specification of the Vacuum}
From the Axiom of Persistence, we have derived three non-negotiable architectural constraints for the substrate of reality: \textbf{Finiteness, Unitarity and Causality}. These constraints constitute the complete specification of System 0. These constraints imply the following requirements:

\begin{enumerate}
    \item ($\Delta E, PM$) \textbf{Finite and Ordered Lattice Field}, required by Finiteness.
    \item ($\Delta I, MI$) \textbf{Positive Definite, Unimodular, Even, and Self-Dual}, required by Unitarity. 
    \item ($G, T$) Capable of a \textbf{Causal Projection} satisfying $\Delta > N$, required by Causality.
\end{enumerate}

With this specification, the physical substrate is no longer an open exploration but a constrained problem: find the \textbf{unique} mathematical object that satisfies all three requirements simultaneously.