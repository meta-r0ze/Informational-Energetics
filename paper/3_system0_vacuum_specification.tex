\section{System 0: The Specification of the Vacuum}
\label{sec:System0_vacuum_specification}

The principle of Informational Energetics, if universal, must apply to the most fundamental layer of reality: the vacuum itself. We construct this specification using the deductive logic of IE. This approach requires analyzing the universe as a physical system subject to the same architectural constraints governing information, energy, and stability that are necessary for any entity to persist. System 0, therefore, represents the formal translation of IE's universal principles into the specific language and constraints of fundamental physics.

\subsection{Derivation A: Persistence Requires Finiteness}
For the vacuum to persist, its definition must be self-consistent. In physics, such inconsistencies manifest as energetic divergences, as exemplified by the Ultraviolet divergences in standard Quantum Field Theory, where assuming continuous spacetime leads to infinite energy density in loop integrals. This is the modern analog of the classical UV catastrophe—the vacuum's zero-point energy formally diverges. IE resolves this by enforcing finiteness as a primary requirement of existence. This corresponds to the \textbf{Capacity} and \textbf{Persistence Margin} pillars.

\begin{itemize}
    \item \textbf{The Logic:} Information is physical and requires energy to store \cite{landauer_irreversibility_1961}. A continuous volume, containing infinite potential information, would therefore possess infinite energy and instantly collapse into a singularity.
    \item \textbf{The Requirement:} To prevent this divergence, the substrate must be \textbf{Discrete}. There must exist a fundamental, indivisible unit of information—a ``pixel'' of reality—that sets a maximum information density for any given volume.
    \item \textbf{Specification:} The universe must be a \textbf{Lattice Field}, not a continuous manifold.
\end{itemize}

\subsection{Derivation B: Persistence Requires Unitarity}
For the vacuum to persist, it must maintain a stable identity over time. This requires that information is perfectly conserved. Since the vacuum \textit{is} the environment, there is no external system to which information can be lost. Therefore, the vacuum must be a perfectly closed and lossless information network. This is the requirement of Unitarity, which maps to the \textbf{Protocol} and \textbf{Identity} pillars.

\begin{itemize}
    \item \textbf{The Logic:} In any information network, from electrical circuits to fiber optics, signal integrity is maximized when the impedance of the source matches the impedance of the load. A mismatch creates irreversible information loss. To create a perfectly lossless network, the system must be perfectly impedance-matched to itself at all points. 
    \item \textbf{The Requirement:} This principle, when applied to a geometric lattice, has a precise mathematical translation. The impedance of a lattice is described by its \textbf{reciprocal lattice} $\Lambda^*$, the Fourier dual that encodes the system's response to external perturbations. For a perfectly reflectionless transfer of information (zero impedance mismatch), the lattice must be \textbf{self-dual}: $\Lambda = \Lambda^*$. Geometrically, this means the lattice looks identical to its own Fourier transform—a rare and highly constrained property. To ensure information density is conserved, it must also be \textbf{Unimodular}.
    \item \textbf{Specification:} The lattice of the vacuum must be \textbf{Unimodular and Self-Dual}.
\end{itemize}

\subsection{Derivation C: Persistence Requires Causality}
For the vacuum to persist, it must not only exist stably but also \textit{evolve} in a well-defined manner. This creates a fundamental information-theoretic challenge: how to project the vast state space of a lattice node onto a single, linear temporal axis without ambiguity. This maps to the \textbf{Governor} and \textbf{Temporal Cost} pillars.

\subsubsection{The Problem of Causal Aliasing}
Let a lattice node possess a total internal state capacity of $N$ distinct information channels. For the universe to evolve, the state of all $N$ channels at one moment must be mapped to the next via a single system-wide update frequency, $\Delta$. By the \textbf{Pigeonhole Principle}, if there are more states ($N$) than time-slots ($\Delta$), at least two distinct states must be mapped to the same moment, creating \textbf{Causal Aliasing}—a catastrophic failure of temporal ordering.

\subsubsection{The Solution: The Generalized Nyquist Limit}
To enforce a strict arrow of time, the system must implement a \textbf{Geometric Governor} that strictly prohibits aliasing. The condition $\Delta = N$ represents a critically saturated channel operating at the Nyquist limit—the minimum sampling rate for perfect signal reconstruction. However, at this critical point, \textbf{any} perturbation (thermal noise, quantum fluctuation) causes aliasing. For a system to persist \textbf{robustly} across cosmic time scales, it must operate with a non-zero margin.

\begin{itemize}
    \item \textbf{The Requirement:} The update frequency must strictly exceed the information bit-depth of the node, providing exactly one unit of causal "breathing room." This is the universe's temporal error-correction margin.
    \item \textbf{Specification:} The projection must satisfy the strict inequality $\Delta > N$. Any configuration violating this is filtered by the Causality Potential ($V_C$), which assigns it an infinite entropic cost.
\end{itemize}

\subsection{Summary: The Architectural Specification of the Vacuum}
From the single axiom of persistence, we have derived three non-negotiable architectural constraints for the substrate of reality. These constraints constitute the complete specification of System 0. The vacuum must be:

\begin{enumerate}
    \item \textbf{Discrete}, to satisfy the requirement of \textbf{Finiteness}.
    \item \textbf{Unimodular and Self-Dual}, to satisfy the requirement of \textbf{Unitarity}.
    \item Capable of a \textbf{Causal Projection} satisfying $\Delta > N$, to satisfy the requirement of \textbf{Causality}.
\end{enumerate}

These are not arbitrary assumptions but the necessary logical consequences of a persistent universe. With this specification defined, the search for the physical substrate is no longer an open exploration but a constrained problem: to find the unique mathematical object that satisfies all three requirements simultaneously. 

In Section~\ref{sec:system1}, we demonstrate that exactly \textbf{one} mathematical structure satisfies all three requirements.