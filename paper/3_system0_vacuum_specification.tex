\section{System 0: The Specification of the Vacuum}
\label{sec:System0_vacuum_specification}

The principle of Informational Energetics, if universal, must apply to the most fundamental layer of reality: the vacuum itself. We construct this specification using the deductive logic of IE. This approach requires analyzing the universe as a physical system subject to the same architectural constraints governing information, energy, and stability that are necessary for any entity to persist. System 0 is the domain-specific application of IE and the field of physics.


\subsection{Derivation A: Persistence Requires Finiteness}
For the vacuum to persist, its definition must be self-consistent. A system defined by infinite properties contains no information and cannot maintain a stable structure. In physics, such inconsistencies manifest as energetic divergences, as exemplified by the Ultraviolet (UV) Catastrophe in standard Quantum Field Theory, where assuming a continuous spacetime leads to an infinite energy density. IE resolves this by enforcing finiteness as a primary requirement of existence. This corresponds to the \textbf{Capacity} and \textbf{Persistence Margin} pillars.

\begin{itemize}
    \item \textbf{The Logic:} Information is physical and requires energy to store (Landauer's Principle). A continuous volume, containing infinite potential information, would therefore possess infinite energy and instantly collapse into a singularity.
    \item \textbf{The Requirement:} To prevent this divergence, the substrate must be \textbf{Discrete}. There must exist a fundamental, indivisible unit of information, a ``pixel'' of reality that sets a maximum information density for any given volume.
    \item \textbf{Specification:} The universe must be a \textbf{Lattice Field}, not a continuous manifold.
\end{itemize}


\subsection{Derivation B: Persistence Requires Unitarity}
For the vacuum to persist, it must maintain a stable identity over time. This requires that information is perfectly conserved; the system cannot lose its own definition. Since the vacuum \textit{is} the environment, there is no external system to which information can be lost. Therefore, the vacuum must be a perfectly closed and lossless information network. This is the requirement of Unitarity, which maps to the \textbf{Protocol} and \textbf{Identity} pillars.

\begin{itemize}
    \item \textbf{The Logic:} In any information network, from electrical circuits to fiber optics, signal integrity is maximized when the impedance of the source matches the impedance of the load. A mismatch creates reflections, scattering, and irreversible information loss. To create a perfectly lossless network (a prerequisite for unitarity) the system must be perfectly impedance-matched to itself at all points. The ``reader'' of information must have the same geometric structure as the ``writer.''
    \item \textbf{The Requirement:} This principle, when applied to a geometric lattice, has a precise mathematical translation. The impedance of a lattice is described by its reciprocal dual ($\Lambda^*$). For a perfect, reflectionless transfer of information, the lattice must be \textbf{Self-Dual} ($\Lambda = \Lambda^*$).
    \item \textbf{Specification:} The lattice of the vacuum must be \textbf{Unimodular and Self-Dual}.
\end{itemize}


\subsection{Derivation C: Persistence Requires Causality}
For the vacuum to persist, it must not only exist stably but also \textit{evolve} in a well-defined manner. A static lattice can store information, but a dynamic universe requires processing, the ordered evolution of state over time. This creates a fundamental information-theoretic challenge: how to project the vast state space of a lattice node onto a single, linear temporal axis without ambiguity or paradox. The system's evolution must be strictly causal. This maps to the \textbf{Governor} and \textbf{Temporal Cost} pillars.

\subsubsection{The Problem of Causal Aliasing}
Let the lattice node possess a total internal state capacity of $N$ distinct information channels (degrees of freedom). For the universe to evolve, the state of all $N$ channels at one moment must be mapped to the next via a single system-wide update frequency, or ``clock cycle,'' $\Delta$.

If the system attempts to map too many distinct states ($N$) onto a timeline that is too short ($\Delta$), a causal collision is inevitable. By the \textbf{Pigeonhole Principle}, if there are more ``pigeons'' ($N$ states) than ``pigeonholes'' ($\Delta$ time-slots), at least two distinct states must be mapped to the same moment in time. This creates \textbf{Causal Aliasing}, a condition where the system cannot distinguish cause from effect, leading to a catastrophic failure of temporal ordering.

\subsubsection{The Solution: The Generalized Nyquist Limit}
To enforce a strict arrow of time, the system must implement a \textbf{Geometric Governor} that strictly prohibits aliasing. This leads to a formal constraint on the projection.

\textbf{Theorem (Generalized Nyquist for Discrete Lattices):} For a bijective (one-to-one) projection from an $N$-channel state space onto a cyclic temporal group $\mathbb{Z}/\Delta\mathbb{Z}$, it is required that:
\begin{equation}
    \Delta \geq N
\end{equation}

\begin{itemize}
    \item \textbf{The Requirement:} The update frequency must equal or exceed the information bit-depth of the node.
    \item \textbf{Specification:} The projection must satisfy $\Delta > N$. Any configuration violating this is filtered by the Causality Potential ($V_C$), which assigns it an infinite entropic cost.
\end{itemize}

\subsection{Summary: The Minimal Architecture of the Vacuum}
The derivations above map to the six function pillars of IE and establish three fundamental and non-negotiable properties of a persistent vacuum substrate.

\begin{enumerate}
    \item \textbf{Discrete ($\Delta E$ \& $PM$)}: Satisfies the requirement of \textbf{Finiteness}.
    \item \textbf{Self-Dual ($\Delta I$ \& $MI$)}: Satisfies the requirement of \textbf{Unitarity}.
    \item \textbf{Causal Projection ($G$ \& $T$)}: Satisfies the requirement of \textbf{Causality}.
\end{enumerate}

In the following section, we analyze the mathematical landscape of lattices to derive the unique geometry that fulfills these requirements.