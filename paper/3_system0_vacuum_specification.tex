\section{System 0: The Specification of the Vacuum}
\label{sec:System0_vacuum_specification}

The principles of Informational Energetics, if truly universal, must apply to the most fundamental layer of reality: the vacuum itself. We now perform the critical act of translation, applying the universal architecture of persistence to this specific domain. This approach treats the universe not as a given, but as a physical system subject to the same architectural constraints governing information, energy, and stability that are necessary for any entity to persist.

This translation of IE's universal principles into the specific language of physics yields three non-negotiable requirements for the substrate of reality:
\begin{itemize}
    \item \textbf{Finiteness:} To prevent energetic and informational divergence.
    \item \textbf{Unitarity:} To ensure the lossless conservation of information.
    \item \textbf{Causality:} To enforce a well-defined temporal evolution.
\end{itemize}

The remainder of this section will derive these three properties in detail, building the complete abstract specification for a persistent universe.

\subsection{Persistence Requires Finiteness}
For the vacuum to persist, its definition must be self-consistent. A system defined by infinite properties contains no information and cannot maintain a stable structure. In physics, such inconsistencies manifest as energetic divergences, as exemplified by the Ultraviolet (UV) Catastrophe in standard Quantum Field Theory, where assuming a continuous spacetime leads to an infinite energy density. IE resolves this by enforcing finiteness as a primary requirement of existence. This corresponds to the \textbf{Capacity} and \textbf{Persistence Margin} pillars.

\begin{itemize}
    \item \textbf{The Logic:} Information is physical and requires energy to store \cite{landauer_irreversibility_1961}. A continuous volume, containing infinite potential information, would therefore possess infinite energy and instantly collapse into a singularity.

    \item \textbf{The Requirement:}
    To prevent Energetic Divergence, the substrate must be \textbf{Discrete}: there must exist a fundamental, indivisible unit of information that sets a maximum density.

    To prevent Algorithmic Divergence, the substrate must be \textbf{Homogeneous}. In Algorithmic Information Theory, the structural complexity of a random graph scales with its size ($K(S) \propto N$). For the universe to be scalable without infinite processing overhead, its Kolmogorov Complexity must be $O(1)$—independent of size. This requires that the local topology be the same regardless of location, minimizing the structural description to a single, repeating rule.

    \item \textbf{Specification:} The universe must be a  \textbf{Lattice Field}, a structure that is both discrete (finite) and translationally invariant (ordered) rather than a continuous manifold or a random graph.
\end{itemize}

\subsection{Persistence Requires Unitarity}
For the vacuum to persist, it must maintain a stable identity over time. This requires that information is perfectly conserved. Since the vacuum \textit{is} the environment, there is no external system to which information can be lost. Therefore, the vacuum must be a perfectly closed and lossless information network. This is the requirement of Unitarity, which maps to the \textbf{Protocol} and \textbf{Identity} pillars.

\begin{itemize}
    \item \textbf{The Logic:} In any information network, signal integrity is maximized when the impedance of the source matches the load. A mismatch creates irreversible information loss. To create a lossless network, the vacuum must be perfectly impedance-matched to itself at all points. 
 
    \item \textbf{The Requirements:}
    To ensure a stable ground state, the \textit{substrate} metric must be \textbf{Positive Definite} (Euclidean).

    \textit{Information-Theoretic Justification}: In a metric space, the norm $|\mathbf{v}|^2 = g_{\mu\nu}v^\mu v^\nu$ represents the information distance from the origin (reference state). For the system to have a well-defined minimum energy configuration, this distance must satisfy:
    \begin{equation}
        |\mathbf{v}|^2 \geq 0 \quad \forall \mathbf{v} \neq 0
    \end{equation}
    
    A metric with negative eigenvalues (Lorentzian) allows $|\mathbf{v}|^2 < 0$ for certain vectors. This implies the ``distance'' from the reference state is imaginary, preventing the definition of a stable ground state (Zero-Point).
    
    Furthermore, a Lorentzian metric admits non-trivial null vectors ($|\mathbf{v}|^2 = 0$ where $\mathbf{v} \neq 0$). This allows for the existence of information states that have zero metric cost. Consequently, the superposition of positive and negative norm states ($|\mathbf{v}_1|^2 + |\mathbf{v}_2|^2 = 0$) would allow for the creation of infinite information density without exceeding the capacity budget. This explicitly violates the \textbf{Finiteness} and \textbf{Capacity} pillars.
    
    Therefore, the substrate metric must be Euclidean (Positive Definite). The observed Lorentzian signature emerges strictly from the causal projection (\cref{sec:spinor}), not the substrate lattice itself.

    For the network to be lossless, the encoding basis must be discrete and orthogonal. Mathematically, this demands that all lattice vector norms are even integers ($\mathbf{v} \cdot \mathbf{v} \in 2\mathbb{Z}$), making the lattice \textbf{Even}.

    \textbf{The Information-Theoretic Necessity:}
    While theoretical physics has long established that self-consistent quantum vacuum must be Even to satisfy \textbf{Spin}\footnote{
    \textbf{Physical Anticipation:} The necessity of evenness for consistent projection is historically rooted in the analysis of modular forms by Serre \cite{serre_course_1973}, who demonstrated that even, self-dual lattices exist only in dimensions divisible by 8. Physically, this constraint is required to support Spin in a Lorentzian manifold as established by Green and Schwarz \cite{green_anomaly_1984}, only specific even lattice structures allow for anomaly-free projections, ensuring the preservation of unitarity and the correct definition of fermionic currents.},
    we derive this constraint here strictly from the \textbf{Identity} pillar of persistence, without assuming a priori physical laws.   

    \textit{The State Enumeration (The Partition Function):}
    To define the Identity of the system, we must enumerate all accessible configurations to establish the bounds of the state space. To satisfy the \textbf{Finiteness} pillar, the system must be spatially bounded. To satisfy the \textbf{Protocol} pillar (Unitarity), it must be closed (no edges where information leaks to an environment). 

    While a sphere satisfies closure, it violates the \textbf{Homogeneity} of a flat Euclidean lattice (requiring curvature). The unique topology that satisfies Finiteness (bounded), Unitarity (closed/periodic), and Homogeneity (locally flat) is the \textbf{Torus} ($T^n$). Consequently, the statistical evolution of the vacuum is defined by modular functions on a torus.
    
    The complete statistical enumeration of states on this geometry is naturally expressed as the Partition Function:
    \begin{equation}
        Z(\beta) = \sum_{\text{states}} e^{-S_{\text{config}}} = \text{Tr}(e^{-\beta H})
    \end{equation}
    where $S_{\text{config}}$ represents the entropic cost (Action) of each configuration. This function is not merely a thermodynamic tool; it is the fundamental generating function for the system's identity, normalizing the probability distribution of all possible histories.

    A persistent system demands a unique, unambiguous equilibrium state. If the system's entropy ($S = -\sum p_i \ln p_i$) depended on the path taken through moduli space (path dependence) rather than its current configuration, the vacuum would lack a stable \textbf{Identity} ($\Delta I$). To enforce this uniqueness, the function that counts these states ($Z$) must be modular invariant, a constraint that is mathematically satisfied only if the underlying lattice is \textbf{Even}.

    \textit{Mathematical Translation}: The complete statistical description of the system is encoded in the partition function $Z(\beta) = \text{Tr}(e^{-\beta H})$ which enumerates accessible states at inverse temperature $\beta$\footnote{We employ the formal apparatus of statistical mechanics (Euclidean Field Theory) as a mathematical tool for state enumeration, not as a claim that the vacuum possesses a literal finite temperature. Here, $\beta$ represents the periodicity of the imaginary time dimension.}. For the equilibrium to be unique, this state-counting function must be single-valued; it cannot depend on the parameterization path taken to reach $Z$.
    
    This uniqueness requirement translates to a mathematical constraint: $Z(\beta)$ must yield consistent state counts under all equivalent parameterizations of the thermal ensemble.

    These parameterizations are the modular transformations:
    \begin{align}
        T: \tau &\to \tau + 1 \quad \text{(Basis shift)}\\
        S: \tau &\to -1/\tau \quad \text{(Dual inversion)}
    \end{align}

    The $T$-transformation corresponds to shifting the lattice basis (coordinate redefinition). The $S$-transformation corresponds to swapping thermal and spatial scales (high-temperature $\leftrightarrow$ low-temperature duality).
    
    \textbf{Crucially, single-valuedness under these transformations is the mathematical expression of path-independence.} Path-independence ensures the vacuum's self-definition is intrinsic, not contingent on arbitrary choices of description—this is the \textbf{Identity} pillar.

    An even lattice ensures this consistency. The Jacobi Theta Function is defined as $\Theta_\Lambda(\tau) = \sum_{\mathbf{v} \in \Lambda} e^{i \pi \tau |\mathbf{v}|^2}$. For Even lattices (where $|\mathbf{v}|^2 = 2n$), the exponent becomes $2\pi i n \tau$, which is manifestly invariant under the shift $\tau \to \tau+1$ (since $e^{2\pi i n} = 1$).
    
    However, any odd-norm vector ($|\mathbf{v}|^2 = 2n+1$) introduces a destructive phase ambiguity. Under the shift $\tau \to \tau+1$, the term transforms as:
    \begin{equation}
        e^{i\pi(\tau+1)(2n+1)} = e^{i\pi\tau(2n+1)} \cdot e^{i\pi(2n+1)} = -e^{i\pi\tau(2n+1)}
    \end{equation}
    This sign inversion ($\Theta \to -\Theta$) renders the partition function multi-valued, violating the Identity pillar. Thus, the requirement for a unique state count strictly mandates that the lattice be \textbf{Even}.

    Having established Evenness, we now complete the modular symmetry required by Identity and derive the requirement that it be \textbf{Self-Dual}. To prevent information loss via destructive interference or basis transformation costs, the encoding operation (write) and the decoding operation (read) must be informationally equivalent. Mathematically, the lattice must be identical to its reciprocal (Fourier dual), making it \textbf{Self-Dual} ($\Lambda = \Lambda^*$). This ensures that the structural metric is its own inverse, allowing for lossless signal propagation without incurring information loss during basis transformations

    This equivalence is mathematically enforced by the modular $S$-transformation ($\tau \to -1/\tau$), which relates the lattice to its reciprocal. By the Poisson Summation formula, the partition function transforms as:
    \begin{equation}
        \Theta_\Lambda(-1/\tau) \propto \frac{1}{\text{vol}(\Lambda)} \Theta_{\Lambda^*}(\tau)
    \end{equation}

    From the Poisson summation formula, the partition function of the dual lattice scales by the inverse volume: $Z_{\Lambda^*} \propto (1/\text{vol}(\Lambda)) Z_{\Lambda}$.
    
    For the Identity to remain invariant under this transformation (so that $Z_{\Lambda} = Z_{\Lambda^*}$), the lattice must be \textbf{Self-Dual} ($\Lambda = \Lambda^*$). This geometric identity strictly enforces \textbf{Unimodularity} ($\text{vol}(\Lambda) = 1$), as the only scalar satisfying $V = 1/V$ is $1$. Any other volume would introduce an irreversible scaling factor during the read/write cycle, violating Unitarity.

    \item \textbf{Specification:} The lattice of the vacuum must be \textbf{Positive Definite, Unimodular, Even, and Self-Dual}.
\end{itemize}

\subsection{Persistence Requires Causality}
For the vacuum to persist, it must not only exist stably but also \textit{evolve} in a well-defined manner. This creates a fundamental information-theoretic challenge: how to project the vast state space of a lattice node onto a single, linear temporal axis without ambiguity. This maps to the \textbf{Governor} and \textbf{Temporal Cost} pillars.

\begin{itemize}
    \item \textbf{The Logic:}
    From the \textbf{Capacity} pillar, we established that any persistent system must have a finite maximum entropy. Let $N$ denote this structural limit—the \textbf{Shannon Channel Capacity} of a single lattice site (the number of distinct internal states available).

    To evolve, the system must map these $N$ parallel state options onto a serial timeline defined by a fundamental integer cycle, the \textbf{Temporal Modulus} ($\Delta$).  
    
    Because the substrate is discrete (\textbf{Finiteness}), time evolution must be quantized; $\Delta \in \mathbb{Z}^+$ is the number of discrete update steps in a fundamental cycle.
    
    By the \textbf{Pigeonhole Principle}, if the timeline's cycle length ($\Delta$) is shorter than the number of state channels ($N$), at least two distinct states must map to the same temporal index. This creates \textbf{Causal Aliasing}—a collision where distinct information states become indistinguishable in time, effectively destroying the system's history.

    \item \textbf{The Requirement:}
    To enforce a strict arrow of time, the system must implement a \textbf{Geometric Governor} that prohibits aliasing.
    
    The condition $\Delta = N$ represents a critically saturated channel (analogous to the Nyquist Limit). However, at this critical point, any perturbation (thermal noise or quantum fluctuation) causes aliasing. For a system to persist \textit{robustly} across cosmic time scales, it must operate with a non-zero margin. Therefore, the temporal container must be strictly larger than the information content it holds.

    \item \textbf{Specification:}
    The projection must satisfy the strict inequality $\Delta > N$. The vacuum is defined by a \textbf{Causal Projection} where the update cycle length exceeds the node capacity.
\end{itemize}

\subsection{Summary: The Architectural Specification of the Vacuum}
From the Axiom of Persistence, we have derived three non-negotiable architectural constraints for the substrate of reality: \textbf{Finiteness, Unitarity and Causality}. These constraints constitute the complete specification of System 0 as these map precisely to the Universal Architecture of Persistence and exhaust the six pillars of persistence, providing a complete specification. These constraints imply the following requirements:

\begin{enumerate}
    \item \textbf{Discrete}, required by Finiteness.
    \item \textbf{Even, Unimodular and Self-Dual}, required by Unitarity.
    \item Capable of a \textbf{Causal Projection} satisfying $\Delta > N$, required by Causality.
\end{enumerate}

With this specification, the physical substrate is no longer an open exploration search but a constrained problem: find the unique mathematical object that satisfies all three requirements simultaneously.