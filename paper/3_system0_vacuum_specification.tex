\section{System 0: The Specification of the Vacuum}
\label{sec:System0_vacuum_specification}

The principles of Informational Energetics, if truly universal, must apply to the most fundamental layer of reality: the vacuum itself. We now perform the critical act of translation, applying the universal architecture of persistence to this specific domain. This approach treats the universe not as a given, but as a physical system subject to the same architectural constraints governing information, energy, and stability that are necessary for any entity to persist.

This translation of IE's universal principles into the specific language of physics yields three non-negotiable requirements for the substrate of reality:
\begin{itemize}
    \item \textbf{Finiteness:} To prevent energetic and informational divergence.
    \item \textbf{Unitarity:} To ensure the lossless conservation of information.
    \item \textbf{Causality:} To enforce a well-defined temporal evolution.
\end{itemize}

The remainder of this section will derive these three properties in detail, building the complete abstract specification for a persistent universe.

\subsection{Persistence Requires Finiteness}
For the vacuum to persist, its definition must be self-consistent. A system defined by infinite properties contains no information and cannot maintain a stable structure. In physics, such inconsistencies manifest as energetic divergences, as exemplified by the Ultraviolet (UV) Catastrophe in standard Quantum Field Theory, where assuming a continuous spacetime leads to an infinite energy density. IE resolves this by enforcing finiteness as a primary requirement of existence. This corresponds to the \textbf{Capacity} and \textbf{Persistence Margin} pillars.

\begin{itemize}
    \item \textbf{The Logic:} Information is physical and requires energy to store \cite{landauer_irreversibility_1961}. A continuous volume, containing infinite potential information, would therefore possess infinite energy and instantly collapse into a singularity.

    \item \textbf{The Requirement:}
    To prevent Energetic Divergence, the substrate must be \textbf{Discrete}: there must exist a fundamental, indivisible unit of information that sets a maximum density.

    To prevent Algorithmic Divergence, the substrate must be \textbf{Homogeneous}. The information required to describe the structure must be independent of its size (constant). This requires that the local topology of a storage cell be the same regardless of location, minimizing the structural description to a single, repeating rule.

    \item \textbf{Specification:} The universe must be a  \textbf{Lattice Field}, a structure that is both discrete (finite) and translationally invariant (ordered) rather than a continuous manifold or a random graph.
\end{itemize}

\subsection{Persistence Requires Unitarity}
For the vacuum to persist, it must maintain a stable identity over time. This requires that information is perfectly conserved. Since the vacuum \textit{is} the environment, there is no external system to which information can be lost. Therefore, the vacuum must be a perfectly closed and lossless information network. This is the requirement of Unitarity, which maps to the \textbf{Protocol} and \textbf{Identity} pillars.

\begin{itemize}
    \item \textbf{The Logic:} In any information network, signal integrity is maximized when the impedance of the source matches the load. A mismatch creates irreversible information loss. To create a lossless network, the vacuum must be perfectly impedance-matched to itself at all points. 
 
    \item \textbf{The Requirements:}
    To ensure a stable ground state, the \textit{substrate} metric must be \textbf{Positive Definite} (Euclidean). A Lorentzian signature at this fundamental level would introduce negative norm states (ghosts), violating the Finiteness pillar. (The observed Lorentzian signature of spacetime is a projection artifact, derived in \cref{sec:spinor}).
    
    To conserve information density, the lattice fundamental domain must have unit volume, making it \textbf{Unimodular}.

    For the network to be lossless, the encoding basis must be discrete and orthogonal. Mathematically, this demands that all lattice vector norms are even integers ($\mathbf{v} \cdot \mathbf{v} \in 2\mathbb{Z}$), making the lattice \textbf{Even}.

    \textbf{The Information-Theoretic Necessity:}
    While theoretical physics has long established that self-consistent quantum vacuum must be Even to satisfy \textbf{Spin}\footnote{
    \textbf{Physical Anticipation:} The necessity of evenness for consistent projection is historically rooted in the analysis of modular forms by Serre \cite{serre_course_1973}, who demonstrated that even, self-dual lattices exist only in dimensions divisible by 8. Physically, this constraint is required to support Spin in a Lorentzian manifold as established by Green and Schwarz \cite{green_anomaly_1984}, only specific even lattice structures allow for anomaly-free projections, ensuring the preservation of unitarity and the correct definition of fermionic currents.},
    we derive this constraint here strictly from the \textbf{Identity} pillar of persistence, without assuming a priori physical laws.   

    A persistent system demands a unique, unambiguous equilibrium state. If the system's entropy ($S = -\sum p_i \ln p_i$) depended on its thermal history rather than its current configuration, the vacuum would lack a stable \textbf{Identity} ($\Delta I$). To enforce this path-independence, the function that counts these states ($Z$) must be modular invariant, a constraint that is mathematically satisfied only if the underlying lattice is \textbf{Even}.
    
    \textit{Mathematical Translation}:The complete statistical description of the system is encoded in the partition function $Z(\beta) = \text{Tr}(e^{-\beta H})$ which enumerates accessible states at inverse temperature $\beta$. For the equilibrium to be unique (history-independent), this state-counting function must be single-valued, it cannot depend on the path taken to reach temperature T.
    
    This uniqueness requirement translates to a mathematical constraint: $Z(\beta)$ must yield consistent state counts under all equivalent parameterizations of the thermal ensemble. 
    
    These parameterizations are the modular transformations ($\tau \to \tau+1$ and $\tau \to -1/\tau$), which correspond to exchanges of temperature and geometric scales. \textbf{Crucially, single-valuedness under these transformations is the mathematical expression of path-independence.}
    
    An even lattice ensures this consistency. The Jacobi Theta Function $\Theta_\Lambda(\tau) = \sum_{\mathbf{v} \in \Lambda} q^{|\mathbf{v}|^2/2}$ has integer coefficients with no half-integer shifts, making it modular invariant. Any odd-norm vector would introduce a branch cut in the state-counting function, breaking this invariance and representing irreversible information loss (Landauer erasure) at the ground state.

    Having established that the lattice must be Even, we now derive the requirement that it be \textbf{Self-Dual}. To prevent information loss via destructive interference or Landauer erasure, the encoding operation (write) and the decoding operation (read) must be informationally equivalent. Mathematically, the lattice must be identical to its reciprocal (Fourier dual), making it \textbf{Self-Dual} ($\Lambda = \Lambda^*$). This ensures that the structural metric is its own inverse, allowing for lossless signal propagation without basis transformation penalties.

    \item \textbf{Specification:} The lattice of the vacuum must be \textbf{Positive Definite, Unimodular, Even, and Self-Dual}.
\end{itemize}

\subsection{Persistence Requires Causality}
For the vacuum to persist, it must not only exist stably but also \textit{evolve} in a well-defined manner. This creates a fundamental information-theoretic challenge: how to project the vast state space of a lattice node onto a single, linear temporal axis without ambiguity. This maps to the \textbf{Governor} and \textbf{Temporal Cost} pillars.

\begin{itemize}
    \item \textbf{The Logic:}
    Consider a lattice node with a total internal state capacity of $N$ distinct information channels (degrees of freedom). To evolve, the system must map these $N$ parallel states onto a serial timeline defined by a fundamental integer cycle, the \textbf{Temporal Modulus} ($\Delta$).
    
    By the \textbf{Pigeonhole Principle}, if the timeline's cycle length ($\Delta$) is shorter than the number of state channels ($N$), at least two distinct states must map to the same temporal index. This creates \textbf{Causal Aliasing}—a collision where distinct information states become indistinguishable in time, effectively destroying the system's history.

    \item \textbf{The Requirement:}
    To enforce a strict arrow of time, the system must implement a \textbf{Geometric Governor} that prohibits aliasing.
    
    The condition $\Delta = N$ represents a critically saturated channel (analogous to the Nyquist Limit). However, at this critical point, any perturbation (thermal noise or quantum fluctuation) causes aliasing. For a system to persist \textit{robustly} across cosmic time scales, it must operate with a non-zero margin. Therefore, the temporal container must be strictly larger than the information content it holds.

    \item \textbf{Specification:}
    The projection must satisfy the strict inequality $\Delta > N$. The vacuum is defined by a \textbf{Causal Projection} where the update cycle length exceeds the node capacity.
\end{itemize}

\subsection{Summary: The Architectural Specification of the Vacuum}
From the Axiom of Persistence, we have derived three non-negotiable architectural constraints for the substrate of reality: \textbf{Finiteness, Unitarity and Causality}. These constraints constitute the complete specification of System 0. These constraints imply the following requirements:

\begin{enumerate}
    \item \textbf{Discrete}, required by Finiteness.
    \item \textbf{Even, Unimodular and Self-Dual}, required by Unitarity.
    \item Capable of a \textbf{Causal Projection} satisfying $\Delta > N$, required by Causality.
\end{enumerate}

With this specification, the physical substrate is no longer an open exploration search but a constrained problem: find the unique mathematical object that satisfies all three requirements simultaneously.