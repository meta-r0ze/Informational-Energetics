\section{System 0: The Specification of the Vacuum}
\label{sec:System0_vacuum_specification}

The principles of Informational Energetics, if universal, must apply to the most fundamental layer of reality: the vacuum itself. We construct this specification using the deductive logic of IE. This approach requires analyzing the universe as a physical system subject to the same architectural constraints governing information, energy, and stability that are necessary for any entity to persist. System 0, therefore, represents the formal translation of IE's universal principles into the specific language and constraints of fundamental physics.

\textit{We now translate the six abstract pillars of persistence into concrete mathematical and physical constraints on the vacuum. Each derivation will correspond to two of the pillars, building up a complete specification for a persistent universe.}

\subsection{Persistence Requires Finiteness}
For the vacuum to persist, its definition must be self-consistent. A system defined by infinite properties contains no information and cannot maintain a stable structure. In physics, such inconsistencies manifest as energetic divergences, as exemplified by the Ultraviolet (UV) Catastrophe in standard Quantum Field Theory, where assuming a continuous spacetime leads to an infinite energy density. IE resolves this by enforcing finiteness as a primary requirement of existence. This corresponds to the \textbf{Capacity} and \textbf{Persistence Margin} pillars.

\begin{itemize}
    \item \textbf{The Logic:} Information is physical and requires energy to store \cite{landauer_irreversibility_1961}. A continuous volume, containing infinite potential information, would therefore possess infinite energy and instantly collapse into a singularity.

    \item \textbf{The Requirement:}
    To prevent Energetic Divergence, the substrate must be \textbf{Discrete}. There must exist a fundamental, indivisible unit of information that sets a maximum density.
    
    To prevent Algorithmic Divergence, the substrate must be \textbf{Homogeneous}. The information required to describe the structure must be independent of its size (constant). This requires that the local topology of a storage cell be the same regardless of location, minimizing the structural description to a single, repeating rule.

    \item \textbf{Specification:} The universe must be a  \textbf{Lattice Field}, a structure that is both discrete (finite) and translationally invariant (ordered) rather than a continuous manifold or a random graph.
\end{itemize}





\subsection{Persistence Requires Unitarity}
For the vacuum to persist, it must maintain a stable identity over time. This requires that information is perfectly conserved. Since the vacuum \textit{is} the environment, there is no external system to which information can be lost. Therefore, the vacuum must be a perfectly closed and lossless information network. This is the requirement of Unitarity, which maps to the \textbf{Protocol} and \textbf{Identity} pillars.

\begin{itemize}
    \item \textbf{The Logic:} In any information network (for example electrical circuits and fiber optics), signal integrity is maximized when the impedance of the source matches the impedance of the load. A mismatch creates irreversible information loss. To create a perfectly lossless network, the system must be perfectly impedance-matched to itself at all points. 

    \item \textbf{The Requirements:}
    To ensure the system has a stable ground state (a `floor' to the energy well), the structural metric must be \textbf{Positive Definite} (Euclidean). A signature containing negative components (Lorentzian) would allow infinite negative energy states, violating the finiteness requirement.
    
    To conserve information density throughout the projection, the lattice fundamental domain must have unit volume, making it \textbf{Unimodular}.

    For the network to be truly non-dissipative, its fundamental energy states must be perfectly discrete, preventing infinitesimal energy leakage. Mathematically, this demands that all lattice vector norms are even integers ($\mathbf{v} \cdot \mathbf{v} \in 2\mathbb{Z}$), making the lattice \textbf{Even}. This geometric constraint also enforces that the substrate's fundamental modes are bosonic (integer-spin), meaning the spin-statistics theorem emerges from the requirement of unitarity.

    To prevent information loss via destructive interference, the lattice must be impedance-matched to itself—the read structure must equal the write structure. Mathematically, the lattice must be identical to its reciprocal (Fourier dual), making it \textbf{Self-Dual} ($\Lambda = \Lambda^*$).

    \item \textbf{Specification:} The lattice of the vacuum must be a \textbf{Positive Definite, Unimodular, Even, and Self-Dual}.
\end{itemize}

\subsection{Persistence Requires Causality}
For the vacuum to persist, it must not only exist stably but also \textit{evolve} in a well-defined manner. This creates a fundamental information-theoretic challenge: how to project the vast state space of a lattice node onto a single, linear temporal axis without ambiguity. This maps to the \textbf{Governor} and \textbf{Temporal Cost} pillars.

\begin{itemize}
    \item \textbf{The Logic:}
    Consider a lattice node with a total internal state capacity of $N$ distinct information channels (degrees of freedom). To evolve, the system must map these $N$ parallel states onto a serial timeline defined by a fundamental integer cycle, the \textbf{Temporal Modulus} ($\Delta$).
    
    By the \textbf{Pigeonhole Principle}, if the timeline's cycle length ($\Delta$) is shorter than the number of state channels ($N$), at least two distinct states must map to the same temporal index. This creates \textbf{Causal Aliasing}—a collision where distinct information states become indistinguishable in time, effectively destroying the system's history.

    \item \textbf{The Requirement:}
    To enforce a strict arrow of time, the system must implement a \textbf{Geometric Governor} that prohibits aliasing.
    
    The condition $\Delta = N$ represents a critically saturated channel (analogous to the Nyquist Limit). However, at this critical point, any perturbation (thermal noise or quantum fluctuation) causes aliasing. For a system to persist \textit{robustly} across cosmic time scales, it must operate with a non-zero margin. Therefore, the temporal container must be strictly larger than the information content it holds.

    \item \textbf{Specification:}
    The projection must satisfy the strict inequality $\Delta > N$. The vacuum is defined by a \textbf{Causal Projection} where the update cycle length exceeds the node capacity. Any configuration violating this is filtered by the Causality Potential ($V_C$), which assigns it an infinite entropic cost.
\end{itemize}



\subsection{Summary: The Architectural Specification of the Vacuum}
From the Axiom of Persistence, we have derived three non-negotiable architectural constraints for the substrate of reality: \textbf{Finiteness, Unitarity and Causality}. These constraints constitute the complete specification of System 0. These constraints imply the following requirements:

\begin{enumerate}
    \item \textbf{Discrete}, required by Finiteness.
    \item \textbf{Even, Unimodular and Self-Dual}, required by Unitarity.
    \item Capable of a \textbf{Causal Projection} satisfying $\Delta > N$, required by Causality.
\end{enumerate}

These are not arbitrary assumptions, but the necessary logical consequences of a persistent universe. With this specification, the search for the physical substrate is no longer an open exploration but a constrained problem: find the unique mathematical object that satisfies all three requirements simultaneously.