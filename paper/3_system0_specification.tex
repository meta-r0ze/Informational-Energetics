\section{System 0: The Specification, The architectural requirements for persistence: Finiteness, Unitarity, Causality}
\label{sec:System0_vacuum_specification}
The principles of Informational Energetics, if truly universal, must apply to the most fundamental layer of reality: the vacuum itself. We now perform the critical act of translation of the universal architecture of persistence to this specific domain to determine the unique configuration that minimizes the Net Entropic Impedance ($Z_{IE}$) to its theoretical floor. In this view, the universe is not a static given, but a self-optimizing physical system subject to the same architectural constraints governing information, energy, and stability necessary for any entity to persist.

The translation of IE's six universal pillars into the specific language of physics yields three non-negotiable requirements for the substrate of reality:
\begin{itemize}
    \item \textbf{Finiteness:} ($CAP, MAR$) To prevent energetic and informational divergence.
    \item \textbf{Unitarity:} ($MAP, PRO$) To ensure the lossless conservation of information.
    \item \textbf{Causality:} ($GOV, TOL$) To enforce a well-defined temporal evolution.
\end{itemize}

The remainder of this section will derive these three properties in detail, building the complete abstract specification for a persistent universe.

\subsection{Persistence Requires \texorpdfstring{$CAP, MAR$}{CAPMAR} via Finiteness}
For the vacuum to persist, its definition must be self-consistent. A system defined by infinite properties contains no information and cannot maintain a stable structure. In physics, such inconsistencies manifest as energetic divergences, as exemplified by the Ultraviolet (UV) Catastrophe in standard Quantum Field Theory. IE resolves this by enforcing finiteness as a primary requirement of existence. This corresponds to the \textbf{Capacity} and \textbf{Persistence Margin} pillars.

\begin{itemize}
    \item \textbf{The Logic:} Information is physical and requires energy to store \cite{landauer_irreversibility_1961}. A continuous volume, containing infinite potential information, would therefore possess infinite energy and instantly collapse into a singularity.

    \item \textbf{The Requirement:}
    To prevent Energetic Divergence, the substrate must be \textbf{Discrete}: there must exist a fundamental, indivisible unit of information that sets a maximum density.
\end{itemize}

To minimize Algorithmic Complexity, the substrate must be \textbf{Homogeneous}. In Algorithmic Information Theory, the structural complexity of a random graph scales with its size ($K(S) \propto N$). For the universe to be scalable without additional processing overhead, its Kolmogorov Complexity must be $O(1)$, independent of size. This requires that the local topology be the same regardless of location, minimizing the structural description to a single, repeating rule.

\begin{itemize}
    \item \textbf{The Requirement:} The universe must be a \textbf{Lattice}, a structure that is both discrete (finite) and translationally invariant (ordered) rather than a continuous manifold or a random graph.
\end{itemize}

\subsection{Persistence Requires \texorpdfstring{$MAP, PRO$}{I MI} via Unitarity}
For the vacuum to persist, it must maintain a stable Map over time. This requires that information is perfectly conserved. Since the vacuum \textit{is} the environment, there is no external system to which information can be lost. Therefore, the vacuum must be a perfectly closed and lossless information network. This is the requirement of Unitarity, which maps to the \textbf{Protocol} and \textbf{Map} pillars.

\subsubsection{Lattice Must Be Positive Definite}
\textit{Information-Theoretic Justification}: In a metric space, the norm $|\mathbf{v}|^2 = g_{\mu\nu}v^\mu v^\nu$ represents the information distance from the origin (reference state). For the system to have a well-defined minimum energy configuration (stable ground state), this distance must satisfy:
\begin{equation}
    |\mathbf{v}|^2 \geq 0 \quad \forall \mathbf{v} \neq 0
\end{equation}
A metric with negative eigenvalues (Lorentzian) allows $|\mathbf{v}|^2 < 0$, implying an imaginary ``distance'' from the reference state and preventing the definition of a stable ground state. Furthermore, a Lorentzian metric admits non-trivial null vectors ($|\mathbf{v}|^2 = 0$ where $\mathbf{v} \neq 0$), allowing for the creation of infinite information density without exceeding the capacity budget ($|\mathbf{v}_1|^2 + |\mathbf{v}_2|^2 = 0$), which violates the \textbf{Finiteness} pillar. Therefore, the \textbf{substrate} must be Euclidean.\footnote{The \textbf{observed} Lorentzian signature of spacetime ($-,+,+,+$) emerges from the causal projection to encode complex chiral states, see \cref{sec:metric_signature}.}

\subsubsection{Lattice Must Be Mappable to a Torus}
To satisfy \textbf{Finiteness}, the system must be spatially bounded. To satisfy \textbf{Protocol} (Unitarity), it must be closed (no edges). While a sphere satisfies closure, a regular lattice cannot be mapped onto curved geometry without defects. The unique topology that satisfies Finiteness (bounded), Unitarity (closed), and Invariance (flat) is the \textbf{Torus} ($T^n$).

Consequently, the statistical evolution of the vacuum is defined by the Partition Function on a torus\footnote{Here, $\beta$ is the formal periodicity parameter of the imaginary time cycle. We use the partition function formalism for state enumeration on a torus, not to claim the vacuum has a literal temperature.}:
\begin{equation}
    Z(\beta) = \sum_{\text{states}} e^{-S_{\text{config}}} = \text{Tr}(e^{-\beta H})
\end{equation}

\subsubsection{Lattice Must Be Even}
A stable Map requires path independence, which in turn requires evenness. A persistent system demands a unique, unambiguous equilibrium state. If the state count $Z$ depended on the path taken through moduli space (parameterization) rather than the configuration itself, the vacuum would lack a stable \textbf{Map}. 

This requires $Z(\beta)$ to be single-valued under the modular transformation $T: \tau \to \tau+1$. The Jacobi Theta Function, $\Theta_\Lambda(\tau) = \sum e^{i \pi \tau |\mathbf{v}|^2}$, counts these states. For \textbf{Even} lattices ($|\mathbf{v}|^2 = 2n$), the exponent $2\pi i n \tau$ is invariant under the shift. However, any odd-norm vector ($|\mathbf{v}|^2 = 2n+1$) introduces a sign inversion ($\Theta \to -\Theta$), rendering the Map multi-valued. Thus, the lattice must be \textbf{Even}.

\subsubsection{Lattice Must Be Self-Dual and Unimodular (Read/Write Symmetry)}
To prevent information loss via destructive interference or Landauer erasure, the encoding operation (write) and the decoding operation (read) must be informationally equivalent. This is enforced by the modular $S$-transformation ($S: \tau \to -1/\tau$), which maps the lattice to its reciprocal (Fourier dual).

By the Poisson Summation formula, the partition function transforms as:
\begin{equation}
    \Theta_\Lambda(-1/\tau) \propto \frac{1}{\text{vol}(\Lambda)} \Theta_{\Lambda^*}(\tau)
\end{equation}
For the Map to remain invariant ($Z_{\Lambda} = Z_{\Lambda^*}$), the lattice must be \textbf{Self-Dual} ($\Lambda = \Lambda^*$). This strictly enforces \textbf{Unimodularity} ($\text{vol}(\Lambda) = 1$), as the only scalar satisfying $V = 1/V$ is $1$. Any other volume introduces an irreversible scaling factor, violating Unitarity.

\textbf{Specification:} The lattice of the vacuum must be \textbf{Positive Definite, Unimodular, Even, and Self-Dual}.

\subsection{Persistence Requires \texorpdfstring{$GOV, TOL$}{GT} via Causality}
\label{sec:Delta_gt_N}
For the vacuum to persist, it must not only exist stably but also \textit{evolve} in a well-defined manner. This creates a fundamental information-theoretic challenge: how to project the vast state space of a lattice node onto a single, linear temporal axis without ambiguity. This maps to the \textbf{Governor} and \textbf{Toll} pillars.

\begin{itemize}
    \item \textbf{The Logic:}
    From the \textbf{Capacity} pillar, we establish that any persistent system must possess a finite state-space cardinality, denoted $N$ (the number of distinct configurations a single node can occupy). 

    To evolve, the system must serialize these $N$ potential states onto a discrete timeline defined by the \textbf{Temporal Modulus} ($\Delta$), representing the number of available temporal slots in one fundamental causal cycle.

    \item \textbf{The Requirement:}
    By the Pigeonhole Principle, if $\Delta \leq N$, multiple distinct states must map to the same temporal coordinate. This results in \textbf{Causal Aliasing}: distinct information states become indistinguishable in time, destroying the system's ability to maintain a well-defined history. 

    \item \textbf{Specification:}
    To enforce a strict arrow of time and satisfy the \textbf{Causality} requirement, the projection must satisfy the \textbf{Persistence Inequality}: $\Delta > N$. The temporal container must strictly exceed the information content it holds.
\end{itemize}

\subsection{Summary: The Architectural Specification of the Vacuum}
From applying IE, we have derived three non-negotiable architectural constraints for the substrate of reality: \textbf{Finiteness, Unitarity, and Causality}. These constraints constitute the complete specification of System 0. These constraints imply the following requirements:

\begin{enumerate}
    \item ($CAP, MAR$) \textbf{Finite Lattice}, required by Finiteness.
    \item ($MAP, PRO$) \textbf{Positive Definite, Unimodular, Even, and Self-Dual}, required by Unitarity. 
    \item ($GOV, TOL$) Any causal system must satisfy a \textbf{Causal Projection}: a serialization of its $N$-dimensional state-space onto a discrete timeline of length $\Delta > N$.
\end{enumerate}

With this specification, the physical substrate is no longer an open exploration but a constrained problem: find the mathematical object that satisfies all three requirements simultaneously.