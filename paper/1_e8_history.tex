\subsection{Theoretical Context}

\subsubsection{The \texorpdfstring{$E_8$}{E8} Lattice: Substrate vs. Algebra}
The exceptional Lie group $E_8$ has long been explored as a candidate for unification due to its status as the largest finite simple symmetry group. Most famously, Lisi proposed embedding the Standard Model directly into the $E_8$ algebra \cite{lisi_exceptionally_2007}. However, Distler and Garibaldi demonstrated that a direct algebraic embedding cannot reproduce the chiral structure of the Standard Model without introducing mirror fermions that are not observed \cite{distler_there_2010}.

We explicitly depart from the algebraic embedding approach. We treat $E_8$ not as the Gauge Algebra (the effective field), but as the \textbf{Geometric Substrate} (the fundamental hardware). By applying Kneser's Theorem \cite{kneser_klassenzahlen_1957}, we derive physics from the \textit{projection} of the $E_8$ lattice onto a 4-dimensional manifold ($E_8 \to D_4 \oplus D_4$). In this framework, chirality emerges strictly from the geometric projection ($E_8 \to D_4$) rather than algebraic embedding, thereby circumventing the Distler-Garibaldi 'No-Go' theorem (detailed in Appendix \ref{sec:ResolutionOfTheDistlerGaribaldi}).. 

Crucially, the lattice defines the \textbf{internal information space}, not a 4D spatial grid. The observable spacetime manifold emerges as the continuous projection of this discrete structure. This ensures that Lorentz Invariance is preserved in the effective field limit, avoiding the preferred-frame violations inherent in naive spatial lattice models.